\documentclass{article}
\usepackage{hyperref}
\usepackage{listings}
\usepackage{graphicx}
\renewcommand{\familydefault}{\sfdefault}
%\usepackage[T1]{fontenc}
\usepackage{textcomp}
\usepackage{makeidx}
\title{PIBASE code guide. ver 2010\\
\includegraphics[scale=0.5]{pibase_blue_web.pdf}}
\author{Fred P. Davis, HHMI-JFRC\\{\tt davisf@janelia.hhmi.org}}
\begin{document}
\maketitle

\begin{abstract}
This document describes the layout of the PIBASE software package. The documentation for all routines is collected here.
\end{abstract}

\section{Overview}

The code used to build and access the database is packaged in (1) a perl library, pibase.pm, (2) a caller script build\_pibase.pl that calls the perl routines in the order necessary to build the database, and (3) a set of auxiliary programs, written in C and perl, that mainly perform PDB file operations.

The goal of this manual is to describe the code in sufficient detail to allow you to design custom queries through the perl interface. Here, I first describe the (1) code layout in pibase.pm, (2) document some of the routines. The routines themselves have inline pod documentation. For now, this document just collects all the inline documentation into a single file.


\begin{verbatim}


perl_api/
|-- LGL.pm
|-- pibase
|   |-- ASTRAL.pm
|   |-- CATH.pm
|   |-- PDB
|   |   |-- chains.pm
|   |   |-- residues.pm
|   |   |-- sec_strx.pm
|   |   `-- subsets.pm
|   |-- PDB.pm
|   |-- PIBASE_core.strx.sql
|   |-- PQS.pm
|   |-- SCOP.pm
|   |-- SGE.pm
|   |-- auxil.pm
|   |-- benchmark.pm
|   |-- build.pm
|   |-- calc
|   |   `-- interfaces.pm
|   |-- create_raw_table_specs.pm
|   |-- data
|   |   |-- access.pm
|   |   |-- calc.pm
|   |   `-- external
|   |       |-- ASTRAL.pm
|   |       `-- PQS.pm
|   |-- data.pm
|   |-- interatomic_contacts.pm
|   |-- kdcontacts.pm
|   |-- modeller.pm
|   |-- pilig.pm
|   |-- raw_table_specs.pm
|   |-- residue_math.pm
|   |-- specs.pm
|   |-- tables_on_disk.pm
|   `-- web.pm
`-- pibase.pm

5 directories, 32 files
\end{verbatim}

%%  Latex generated from POD in document /groups/eddy/home/davisf/work/pibase/pibase_work/trunk/docs/../src/perl_api/LGL.pm
%%  Using the perl module Pod::LaTeX
%%  Converted on Sun Sep 19 04:36:38 2010

%%  Preamble supplied by user.

\clearpage
\section{LGL.pm\label{LGL_pm}\index{LGL.pm}}


Perl interface to Alex Adai's Large Graph Layout (LGL) program

\subsection*{DESCRIPTION\label{LGL_pm_DESCRIPTION}\index{LGL pm!DESCRIPTION}}


The LGL.pm package provides a perl interface to LGL so that edge/node lists can
be converted into postscript or png images with user-configurable edge and node
colors, shapes, and sizes.

\subsection*{AUTHOR\label{LGL_pm_AUTHOR}\index{LGL pm!AUTHOR}}


Fred P. Davis, HHMI-JFRC (davisf@janelia.hhmi.org)

\subsection*{SYNOPSIS\label{LGL_pm_SYNOPSIS}\index{LGL pm!SYNOPSIS}}
\subsubsection*{Example 1\label{LGL_pm_Example_1}\index{LGL pm!Example 1}}
\begin{verbatim}
   my $edges = {
      a => { 'b' => 1} ,
      a => { 'd' => 1} ,
      a => { 'e' => 1} ,
      b => { 'c' => 1} ,
      c => { 'e' => 1} ,
   } ;
   my $coords = LGL::edges2coords({edges => $edges}) ;
   coords2ps({
      edges => $edges,
      coords => $coords,
   }) ;
\end{verbatim}
\subsubsection*{Example 2\label{LGL_pm_Example_2}\index{LGL pm!Example 2}}
\begin{verbatim}
   my $edges ;
   $edges->{a}->{b} = 1 ;
   $edges->{a}->{c} = 1 ;
   $edges->{b}->{d} = 1 ;
\end{verbatim}
\begin{verbatim}
   my $ncols;
   $ncols->{"a"} = "black" ;
   $ncols->{"b"} = "yellow";
   $ncols->{"c"} = "cyan";
   $ncols->{"d"} = "red";
\end{verbatim}
\begin{verbatim}
   my $ecols ;
   $ecols->{a}->{b} = "black" ;
   $ecols->{a}->{c} = "purple" ;
   $ecols->{b}->{d} = "brown" ;
\end{verbatim}
\begin{verbatim}
   my $coords = LGL::edges2coords({edges => $edges}) ;
   LGL::coords2ps({coords => $coords,
                   edges => $edges,
                   mag => 100,
                   ethick => 1,
                   nrad => 5,
                   nshape => 'fbox',
                   ncol => "0.5 0.3 0.2",
                   ncols => $ncols,
                   ecols => $ecols,
                   ecol => "0.2 0.7 0.4"}) ;
\end{verbatim}
\subsection*{SUBROUTINES\label{LGL_pm_SUBROUTINES}\index{LGL pm!SUBROUTINES}}
\subsubsection*{edges2coords()\label{LGL_pm_edges2coords_}\index{LGL pm!edges2coords()}}
\begin{verbatim}
   Title:       edges2coords()
   Function:    Performs graph layout using LGL, returning coordinate info
   Args:        $_->{edges}->{node1}->{node2} ;
   Returns:     $_->{node} = [$x, $y] ;
\end{verbatim}
\subsubsection*{\_make\_lglconfig()\label{LGL_pm__make_lglconfig_}\index{LGL pm!\ make\ lglconfig()}}
\begin{verbatim}
   Title:       _make_lglconfig
   Function:    Makes a config file for an LGL run.
   Returns:     nothing
   Args:        $_->{cnfg_fh} (file handle to display the configuration file to)
\end{verbatim}
\subsubsection*{coords2pngmap()\label{LGL_pm_coords2pngmap_}\index{LGL pm!coords2pngmap()}}
\begin{verbatim}
   Title:       coords2pngmap
   Function:    Creates a client-side png image map given coordinate and
                  edge information. 
   Returns:     nothing
   Args:
      - $_->{ncol}      node color
      - $_->{ecol}      edge color
      - $_->{estyle}    edge style (solid, dashed)
      - $_->{nshape}    node shape (triangle, square, circle)
      - $_->{nrad}      node radius
      - $_->{ethick}    edge thickness
      - $_->{mag}       magnification parameter
      - $_->{coords}    node coordinate information
      - $_->{edges}->{n1}->{n2} edge list (nhash)
      - $_->{map_fh}    output imagemap file name
      - $_->{nodenames} node names
      - $_->{nodeurls}  node URLs
\end{verbatim}
\subsubsection*{coords2png()\label{LGL_pm_coords2png_}\index{LGL pm!coords2png()}}
\begin{verbatim}
   Title:       coords2png
   Function:    Creates a png image of a graph, given layout info
   Returns:     GD::image object
   Args:
      $_->{coords}      node coordinate information
      $_->{edges}->{n1}->{n2}   edge list (nhash)
      $_->{ncol}        default node color
      $_->{ncols}->{n}  node-specific color
      $_->{ecol}        default edge color
      $_->{ecols}->{n1}->{n2}   edge-specific color
      $_->{estyle}      default edge style - solid or dashed
      $_->{estyles}->{n1}->{n2} edge-specific style
      $_->{nshape}      node shape ([f]{triangle|box|circle})
      $_->{nshapes}->{n}        node-specific shape
      $_->{nrad}        node radius
      $_->{ethick}      edge thickness
      $_->{mag} magnification parameter
\end{verbatim}
\subsubsection*{coords2ps()\label{LGL_pm_coords2ps_}\index{LGL pm!coords2ps()}}
\begin{verbatim}
   Title:       coords2ps
   Function:    Creates a postscript image of a graph, given layout info
   Returns:     nothing
   Args:
      $_->{coords}      node coordinate information
      $_->{edges}->{n1}->{n2}   edge list (nhash)
      $_->{ncol}        default node color
      $_->{ncols}->{n}  node-specific color
      $_->{ecol}        default edge color
      $_->{ecols}->{n1}->{n2}   edge-specific color
      $_->{estyle}      default edge style - solid or dashed
      $_->{estyles}->{n1}->{n2} edge-specific style
      $_->{nshape}      node shape ([f]{triangle|box|circle})
      $_->{nshapes}->{n}        node-specific shape
      $_->{nrad}        node radius
      $_->{ethick}      edge thickness
      $_->{mag} magnification parameter
\end{verbatim}
\subsubsection*{get\_color2rgb()\label{LGL_pm_get_color2rgb_}\index{LGL pm!get\ color2rgb()}}
\begin{verbatim}
   Title:       get_color2rgb()
   Function:    Returns a hash pointing from color name to rgb (0-1) values
   Args:        Nothing
   Returns:     $_->{color} = "$r $g $b" ;
\end{verbatim}
\clearpage
\section{pibase.pm\label{pibase_pm}\index{pibase.pm}}


Perl interface to the pibase database

\subsection*{DESCRIPTION\label{pibase_pm_DESCRIPTION}\index{pibase pm!DESCRIPTION}}


The pibase.pm perl library contains subroutines used to build and access the
PIBASE database.

\subsection*{AUTHOR\label{pibase_pm_AUTHOR}\index{pibase pm!AUTHOR}}


Fred P. Davis, HHMI-JFRC (davisf@janelia.hhmi.org)

\subsection*{SUBROUTINES\label{pibase_pm_SUBROUTINES}\index{pibase pm!SUBROUTINES}}
\subsubsection*{dbname()\label{pibase_pm_dbname_}\index{pibase pm!dbname()}}
\begin{verbatim}
   Title:       dbname()
   Function:    gets the name of the pibase database
   Args:        pibase $specs data structure
   Returns:     returns the name of the pibase mysql database
\end{verbatim}
\subsubsection*{get\_specs()\label{pibase_pm_get_specs_}\index{pibase pm!get\ specs()}}
\begin{verbatim}
   Title:       get_specs()
   Function:    gives pibase specifications
   Args:        pibase $specs data structure
   Returns:     returns completed pibase data specifications
\end{verbatim}
\subsubsection*{connect\_pibase(\$dbspecs)\label{pibase_pm_connect_pibase_dbspecs_}\index{pibase pm!connect\ pibase(\$dbspecs)}}
\begin{verbatim}
   Title:       connect_pibase()
   Function:    Connects to the pibase database.
   Args:        $_->{db}        database name
                $_->{user}      user name
                $_->{pass}      password
   Returns:     DBI database handle to pibaes
\end{verbatim}
\subsubsection*{connect\_tod(\$dbspecs)\label{pibase_pm_connect_tod_dbspecs_}\index{pibase pm!connect\ tod(\$dbspecs)}}
\begin{verbatim}
   Title:       connect_tod()
   Function:    Connects to pibase tables on disk (POD).
   Args:        $_[0] = tablename
                $_[1] = dbspecs
                $_[2] = tod_dir = location of tables on disk
\end{verbatim}
\begin{verbatim}
   Returns:     DBI database handle to table on disk
\end{verbatim}
\subsubsection*{rawselect\_tod()\label{pibase_pm_rawselect_tod_}\index{pibase pm!rawselect\ tod()}}
\begin{verbatim}
   Title:       rawselect_tod()
   Function:    performs basic SELECT statement queries on a table on disk
   Args:        $_[0] = SQL SELECT-like statement
                $_[1] = fullfile
\end{verbatim}
\subsubsection*{connect\_metatod()\label{pibase_pm_connect_metatod_}\index{pibase pm!connect\ metatod()}}
\begin{verbatim}
   Title:       connect_metatod()
   Function:    performs basic SELECT statement queries on a meta-table on disk
   Args:        $_[0] = filename
                $_[1] = tablename
                $_[2] = pibase db specs
   Returns:     dbh DBI:AnyData database handle
\end{verbatim}
\subsubsection*{rawselect\_metatod()\label{pibase_pm_rawselect_metatod_}\index{pibase pm!rawselect\ metatod()}}
\begin{verbatim}
   Title:       rawselect_metatod()
   Function:    selects specified fields from a table-on-disk.
      Note: WHERE clause does not work, this command just recognizes
            the field names and returns the appropriate columns.
   Args:        $_[0] = filename
                $_[1] = SELECT sql command
   Returns:     array of query results
\end{verbatim}
\subsubsection*{sid\_2\_domdir()\label{pibase_pm_sid_2_domdir_}\index{pibase pm!sid\ 2\ domdir()}}
\begin{verbatim}
   Title:       sid_2_domdir()
   Function:    returns the directory name where the PDB file of the
      specified domain resides.
   Args:        $_ = subset_id
   Returns:     directory name
\end{verbatim}
\subsubsection*{complete\_pibase\_specs(specs)\label{pibase_pm_complete_pibase_specs_specs_}\index{pibase pm!complete\ pibase\ specs(specs)}}
\begin{verbatim}
   Title:       complete_pibase_specs
   Function:    Fills in blanks in specs with default values.
   Input:       $_ = specs hashref
                $_->{db} = database_name
                $_->{user} = user name
                $_->{pass} = password
   Return:      specs - hashref
\end{verbatim}
\subsubsection*{load\_bdp\_ids(\$dbh, @results\_type)\label{pibase_pm_load_bdp_ids_dbh_results_type_}\index{pibase pm!load\ bdp\ ids(\$dbh, @results\ type)}}
\begin{verbatim}
   Name: load_bdp_ids()
   Function:    Returns bdp_id and depending on results_type
                specified, its relation to bdp_path and pdb_id
                in a variety of forms.
   Return:      results
   Args:        $_[0] = DBI dbh handle
                $_[1] = results type
\end{verbatim}
\begin{itemize}

\item path\_2\_bdp\_id (hash) [default]
\item bdp\_id\_2\_path (hash)
\item bdp\_id\_2\_pdb\_id (hash)
\item bdp\_id\_2\_raw\_pdb (hash)
\item pdb\_id\_2\_bdp\_id (hash)
\item bdp\_id (array)\end{itemize}
\subsubsection*{todload\_bdp\_ids(@results\_type)\label{pibase_pm_todload_bdp_ids_results_type_}\index{pibase pm!todload\ bdp\ ids(@results\ type)}}
\begin{verbatim}
   Name: todload_bdp_ids()
   Function:    Returns bdp_id and depending on results_type
                specified, its relation to bdp_path and pdb_id
                in a variety of forms.
      Analogous to load_bdp_ids() with tables-on-disk instead of DBI
   Return:      query results
   Args:        $_[0] = DBI dbh handle
                $_[1] = results type
\end{verbatim}
\begin{itemize}

\item path\_2\_bdp\_id (hash) [default]
\item bdp\_id\_2\_path (hash)
\item bdp\_id\_2\_pdb\_id (hash)
\item bdp\_id\_2\_raw\_pdb (hash)
\item pdb\_id\_2\_bdp\_id (hash)
\item bdp\_id (array)\end{itemize}
\subsubsection*{mysql\_fetchcols(dbh, query)\label{pibase_pm_mysql_fetchcols_dbh_query_}\index{pibase pm!mysql\ fetchcols(dbh, query)}}
\begin{verbatim}
   Function:    Processes an n column query and returns a list of array references, where each array reference holds all the valeus for a given column.
   Args:        $_[0] = dbh - DBI database handle
                $_[1] = SQL select query
\end{verbatim}
\begin{verbatim}
   Returns:     @_ - list of arrayref
                $a[i]->[j] ith column, jth row
\end{verbatim}
\subsubsection*{mysql\_hashindload(dbh, query)\label{pibase_pm_mysql_hashindload_dbh_query_}\index{pibase pm!mysql\ hashindload(dbh, query)}}
\begin{verbatim}
   Name:        mysql_hashindload() ;
   Function:    Processes a 1 column query and returns a hash pointing from
                  column1 values to row number.
   Args:        $_[0] = dbh - DBI database handle
                $_[1] = SQL SELECT query
                $_[2] = substitutor for undefined value
   Returns:     $_ = hashref
                $a->{col1} = row number
\end{verbatim}
\subsubsection*{array2hash(arrayref, undef\_sub)\label{pibase_pm_array2hash_arrayref_undef_sub_}\index{pibase pm!array2hash(arrayref, undef\ sub)}}
\begin{verbatim}
   Name:        array2hash() ;
   Function:    Takes an array reference and returns a hashref with value,
                index pairs.
   Args:        $_[0] = array references
                $_[1] = undefined substitution - if the cell contains an
                  undefined value, use this value as the hash key
   Returns:     $_ = hashref
                $a->{value} = row number
\end{verbatim}
\subsubsection*{replace\_undefs(arrayref, undef\_sub)\label{pibase_pm_replace_undefs_arrayref_undef_sub_}\index{pibase pm!replace\ undefs(arrayref, undef\ sub)}}
\begin{verbatim}
   Name:        array2hash() ;
   Function:    Takes an array reference and replaces (inplace) undefined
                  values to a specified substitution value.
   Args:        $_[0] = array references
                $_[1] = undefined substitution - if the cell contains an
                  undefined value, replace with this value
   Returns:     $_ = array reference
\end{verbatim}
\subsubsection*{replace\_undefs\_blanks(arrayref, undef\_sub)\label{pibase_pm_replace_undefs_blanks_arrayref_undef_sub_}\index{pibase pm!replace\ undefs\ blanks(arrayref, undef\ sub)}}
\begin{verbatim}
   Name:        array2hash() ;
   Function:    Takes an array reference and replaces (inplace) undefined and
                  blank values to a specified substitution value.
   Args:        $_[0] = array references
                $_[1] = undefined/blank substitution - if the cell
                  contains an undefined or blank value, replace with this value
   Returns:     $_ = array reference
\end{verbatim}
\subsubsection*{replace\_char(arrayref, target, replacement)\label{pibase_pm_replace_char_arrayref_target_replacement_}\index{pibase pm!replace\ char(arrayref, target, replacement)}}
\begin{verbatim}
   Name:        array2hash() ;
   Function:    Takes an array reference and replaces (inplace) target values to a specified substitution value.
   Args:        $_[0] = array reference
                $_[1] = target value
                $_[2] = substitution value
   Returns:     $_ = array reference
\end{verbatim}
\subsubsection*{mysql\_hashload(dbh, query)\label{pibase_pm_mysql_hashload_dbh_query_}\index{pibase pm!mysql\ hashload(dbh, query)}}
\begin{verbatim}
   Name:        mysql_hashload()
   Function:    Processes a 2 column query and returns a hash pointing from
                  column1 values to column2 values (use for 1:1 relationships)
   Args:        $_[0] = dbh - DBI database handle
                $_[1] = SQL SELECT command
   Returns:     $a - hashref
                $a->{col1} = col2
\end{verbatim}
\subsubsection*{mysql\_hash2load(dbh, query)\label{pibase_pm_mysql_hash2load_dbh_query_}\index{pibase pm!mysql\ hash2load(dbh, query)}}
\begin{verbatim}
   Name:        mysql_hash2load()
   Function:    Processes a 3 column query and returns a hash pointing from
                  column1 values to column2 to column3 values
                  (use for 1:1:1 relationships)
   Args:        $_[0] = dbh - DBI database handle
                $_[1] = SQL SELECT command
   Returns:     $a - hashref
                $a->{col1}->{col2} = col3 ;
\end{verbatim}
\subsubsection*{mysql\_hasharrload(dbh, query, undef\_subs)\label{pibase_pm_mysql_hasharrload_dbh_query_undef_subs_}\index{pibase pm!mysql\ hasharrload(dbh, query, undef\ subs)}}
\begin{verbatim}
   Name:        mysql_hashload()
   Function:    Processes a 2 column query and returns a hash pointing from
                  column1 values to an array of column2 values
                  (use for 1:n relationships)
   Args:        $_[0] dbh - DBI database handle
                $_[1] query - SQL format
                $_[2] undefined value substitutors - list
   Returns:     $a - hashref to arrayrefs
   $a->{col1} = [ col2_1, col2_2,... col2_n ]
\end{verbatim}
\subsubsection*{mysql\_singleval(dbh, query)\label{pibase_pm_mysql_singleval_dbh_query_}\index{pibase pm!mysql\ singleval(dbh, query)}}
\begin{verbatim}
   Function:    Processes a 1 column, 1 row query and returns a scalar
                  containing the value.
   Args:        $_[0] = dbh - DBI database handle
                $_[1] = query - SQL format
   Return:      $a - scalar
                $a = "result"
\end{verbatim}
\subsubsection*{timestamp()\label{pibase_pm_timestamp_}\index{pibase pm!timestamp()}}
\begin{verbatim}
   Function:    Returns a timestamp
   Args:        none
   Return:      $_[0] = timestamp: <4-digit YEAR><2-digit MONTH><2-digit DAY>_
                  <2-digit HOUR><2-digit MINUTE>
\end{verbatim}
\subsubsection*{get\_current\_date\_mysql()\label{pibase_pm_get_current_date_mysql_}\index{pibase pm!get\ current\ date\ mysql()}}
\begin{verbatim}
   Function:    Returns current date in YYYY-MM-DD mysql format
   Args:        none
   Return:      $_[0] = date: <4-digit YEAR>-<2-digit MONTH>-<2-digit DAY>
\end{verbatim}
\subsubsection*{timestampsec()\label{pibase_pm_timestampsec_}\index{pibase pm!timestampsec()}}
\begin{verbatim}
   Function:    Returns a second-resolution timestamp
   Args:        none
   Return:      timestamp: <4-digit YEAR><2-digit MONTH><2-digit DAY>_
                  <2-digit HOUR><2-digit MINUTE><2-digit SECOND>
\end{verbatim}
\subsubsection*{mysqlimport(file,dbspecs)\label{pibase_pm_mysqlimport_file_dbspecs_}\index{pibase pm!mysqlimport(file,dbspecs)}}
\begin{verbatim}
   Function:    load a file into a mysql database using system(mysqlimport)
   Return:      $_[0] = records imported
                $_[1] = records deleted
                $_[2] = records skipped
                $_[3] = number of warnings
\end{verbatim}
\begin{verbatim}
   Args:        $_[0] = filename
                $_[1] = database specs - hashref
\end{verbatim}
\begin{itemize}

\item db =$>$ database name
\item user =$>$ user name
\item pass =$>$ password\end{itemize}
\subsubsection*{mysql\_runcom(dbh, query, vaues)\label{pibase_pm_mysql_runcom_dbh_query_vaues_}\index{pibase pm!mysql\ runcom(dbh, query, vaues)}}
\begin{verbatim}
   Function:    Runs an non-SELECT SQL statement
   Return:      none
   Args:        $_[0] = dbh - DBI database handle
                $_[1] = SQL query command or DBI statement handle
                $_[2] = values- arrayref that hold values for '?' holders
                        in DBI statement handle
\end{verbatim}
\subsubsection*{mysql\_createtable(dbh, tablename, spec)\label{pibase_pm_mysql_createtable_dbh_tablename_spec_}\index{pibase pm!mysql\ createtable(dbh, tablename, spec)}}
\begin{verbatim}
   Function:    Creates a table (NOTE: if table already exists, drops it first).
   Return:      none
   Args:        $_[0] = dbh - DBI database handle
                $_[1] = table name
                $_[2] = spec - SQL DDL string
\end{verbatim}
\begin{verbatim}
   Example uasge:
\end{verbatim}
\begin{verbatim}
      pibase::mysql_createtable($dbh,
         $tables->{interface_contacts}->{name},
         $tables->{interface_contacts}->{spec}) ;
\end{verbatim}
\begin{verbatim}
      pibase::mysql_runcom($dbh,
         "REPLACE INTO $tables->{interface_contacts}->{meta} ".
         "( bdp_id, table_name) values($bdp_id, ".
         "\"$tables->{interface_contacts}->{name}\")") ;
\end{verbatim}
\subsubsection*{mysql\_commandline\_query(dbh, query, vaues)\label{pibase_pm_mysql_commandline_query_dbh_query_vaues_}\index{pibase pm!mysql\ commandline\ query(dbh, query, vaues)}}
\begin{verbatim}
   Function:    Runs an SQL SELECT statement on the command line,
                optionally performs a sort (on command line), and 
                displays the output to a specified file
   Return:      none
   Args:        $->{db_name} = database name
                $->{sql} = SQL query
                $->{out_fn} = file to display results to
                $->{post_sort} = optional specify field to sort
                $->{sort_order} = optional sort order (defaults ASC)
\end{verbatim}
\subsubsection*{locate\_binaries()\label{pibase_pm_locate_binaries_}\index{pibase pm!locate\ binaries()}}
\begin{verbatim}
   Function:    Returns location of binaries used in PIBASE associated activities
   Return:      $_->{program} = program location.
      perl, zcat, rigor, subset_extractor, altloc_check
   Args:        none
\end{verbatim}
\subsubsection*{safe\_move()\label{pibase_pm_safe_move_}\index{pibase pm!safe\ move()}}
\begin{verbatim}
   Function:    Safely move a file to a directory (using File::Copy::move),
                  retries 14 times, and prints an error if it didnt work
   Return:      nothing
   Args:        $_[0] = source filename
                $_[1] = target directory
\end{verbatim}
\subsubsection*{safe\_copy()\label{pibase_pm_safe_copy_}\index{pibase pm!safe\ copy()}}
\begin{verbatim}
   Function:    Safely copy a file to a directory (using File::Copy::copy),
                  retries 14 times, and prints an error if it didnt work
   Return:      nothing
   Args:        $_[0] = source filename
                $_[1] = target directory
\end{verbatim}
\clearpage
\section{pibase.pm\label{pibase_pm}\index{pibase.pm}}


Perl interface to the pibase database

\subsection*{DESCRIPTION\label{pibase_pm_DESCRIPTION}\index{pibase pm!DESCRIPTION}}


The pibase.pm perl library contains subroutines used to build and access the
PIBASE database.

\subsection*{AUTHOR\label{pibase_pm_AUTHOR}\index{pibase pm!AUTHOR}}


Fred P. Davis, HHMI-JFRC (davisf@janelia.hhmi.org)

\subsection*{SUBROUTINES\label{pibase_pm_SUBROUTINES}\index{pibase pm!SUBROUTINES}}
\subsubsection*{dbname()\label{pibase_pm_dbname_}\index{pibase pm!dbname()}}
\begin{verbatim}
   Title:       dbname()
   Function:    gets the name of the pibase database
   Args:        pibase $specs data structure
   Returns:     returns the name of the pibase mysql database
\end{verbatim}
\subsubsection*{get\_specs()\label{pibase_pm_get_specs_}\index{pibase pm!get\ specs()}}
\begin{verbatim}
   Title:       get_specs()
   Function:    gives pibase specifications
   Args:        pibase $specs data structure
   Returns:     returns completed pibase data specifications
\end{verbatim}
\subsubsection*{connect\_pibase(\$dbspecs)\label{pibase_pm_connect_pibase_dbspecs_}\index{pibase pm!connect\ pibase(\$dbspecs)}}
\begin{verbatim}
   Title:       connect_pibase()
   Function:    Connects to the pibase database.
   Args:        $_->{db}        database name
                $_->{user}      user name
                $_->{pass}      password
   Returns:     DBI database handle to pibaes
\end{verbatim}
\subsubsection*{connect\_tod(\$dbspecs)\label{pibase_pm_connect_tod_dbspecs_}\index{pibase pm!connect\ tod(\$dbspecs)}}
\begin{verbatim}
   Title:       connect_tod()
   Function:    Connects to pibase tables on disk (POD).
   Args:        $_[0] = tablename
                $_[1] = dbspecs
                $_[2] = tod_dir = location of tables on disk
\end{verbatim}
\begin{verbatim}
   Returns:     DBI database handle to table on disk
\end{verbatim}
\subsubsection*{rawselect\_tod()\label{pibase_pm_rawselect_tod_}\index{pibase pm!rawselect\ tod()}}
\begin{verbatim}
   Title:       rawselect_tod()
   Function:    performs basic SELECT statement queries on a table on disk
   Args:        $_[0] = SQL SELECT-like statement
                $_[1] = fullfile
\end{verbatim}
\subsubsection*{connect\_metatod()\label{pibase_pm_connect_metatod_}\index{pibase pm!connect\ metatod()}}
\begin{verbatim}
   Title:       connect_metatod()
   Function:    performs basic SELECT statement queries on a meta-table on disk
   Args:        $_[0] = filename
                $_[1] = tablename
                $_[2] = pibase db specs
   Returns:     dbh DBI:AnyData database handle
\end{verbatim}
\subsubsection*{rawselect\_metatod()\label{pibase_pm_rawselect_metatod_}\index{pibase pm!rawselect\ metatod()}}
\begin{verbatim}
   Title:       rawselect_metatod()
   Function:    selects specified fields from a table-on-disk.
      Note: WHERE clause does not work, this command just recognizes
            the field names and returns the appropriate columns.
   Args:        $_[0] = filename
                $_[1] = SELECT sql command
   Returns:     array of query results
\end{verbatim}
\subsubsection*{sid\_2\_domdir()\label{pibase_pm_sid_2_domdir_}\index{pibase pm!sid\ 2\ domdir()}}
\begin{verbatim}
   Title:       sid_2_domdir()
   Function:    returns the directory name where the PDB file of the
      specified domain resides.
   Args:        $_ = subset_id
   Returns:     directory name
\end{verbatim}
\subsubsection*{complete\_pibase\_specs(specs)\label{pibase_pm_complete_pibase_specs_specs_}\index{pibase pm!complete\ pibase\ specs(specs)}}
\begin{verbatim}
   Title:       complete_pibase_specs
   Function:    Fills in blanks in specs with default values.
   Input:       $_ = specs hashref
                $_->{db} = database_name
                $_->{user} = user name
                $_->{pass} = password
   Return:      specs - hashref
\end{verbatim}
\subsubsection*{load\_bdp\_ids(\$dbh, @results\_type)\label{pibase_pm_load_bdp_ids_dbh_results_type_}\index{pibase pm!load\ bdp\ ids(\$dbh, @results\ type)}}
\begin{verbatim}
   Name: load_bdp_ids()
   Function:    Returns bdp_id and depending on results_type
                specified, its relation to bdp_path and pdb_id
                in a variety of forms.
   Return:      results
   Args:        $_[0] = DBI dbh handle
                $_[1] = results type
\end{verbatim}
\begin{itemize}

\item path\_2\_bdp\_id (hash) [default]
\item bdp\_id\_2\_path (hash)
\item bdp\_id\_2\_pdb\_id (hash)
\item bdp\_id\_2\_raw\_pdb (hash)
\item pdb\_id\_2\_bdp\_id (hash)
\item bdp\_id (array)\end{itemize}
\subsubsection*{todload\_bdp\_ids(@results\_type)\label{pibase_pm_todload_bdp_ids_results_type_}\index{pibase pm!todload\ bdp\ ids(@results\ type)}}
\begin{verbatim}
   Name: todload_bdp_ids()
   Function:    Returns bdp_id and depending on results_type
                specified, its relation to bdp_path and pdb_id
                in a variety of forms.
      Analogous to load_bdp_ids() with tables-on-disk instead of DBI
   Return:      query results
   Args:        $_[0] = DBI dbh handle
                $_[1] = results type
\end{verbatim}
\begin{itemize}

\item path\_2\_bdp\_id (hash) [default]
\item bdp\_id\_2\_path (hash)
\item bdp\_id\_2\_pdb\_id (hash)
\item bdp\_id\_2\_raw\_pdb (hash)
\item pdb\_id\_2\_bdp\_id (hash)
\item bdp\_id (array)\end{itemize}
\subsubsection*{mysql\_fetchcols(dbh, query)\label{pibase_pm_mysql_fetchcols_dbh_query_}\index{pibase pm!mysql\ fetchcols(dbh, query)}}
\begin{verbatim}
   Function:    Processes an n column query and returns a list of array references, where each array reference holds all the valeus for a given column.
   Args:        $_[0] = dbh - DBI database handle
                $_[1] = SQL select query
\end{verbatim}
\begin{verbatim}
   Returns:     @_ - list of arrayref
                $a[i]->[j] ith column, jth row
\end{verbatim}
\subsubsection*{mysql\_hashindload(dbh, query)\label{pibase_pm_mysql_hashindload_dbh_query_}\index{pibase pm!mysql\ hashindload(dbh, query)}}
\begin{verbatim}
   Name:        mysql_hashindload() ;
   Function:    Processes a 1 column query and returns a hash pointing from
                  column1 values to row number.
   Args:        $_[0] = dbh - DBI database handle
                $_[1] = SQL SELECT query
                $_[2] = substitutor for undefined value
   Returns:     $_ = hashref
                $a->{col1} = row number
\end{verbatim}
\subsubsection*{array2hash(arrayref, undef\_sub)\label{pibase_pm_array2hash_arrayref_undef_sub_}\index{pibase pm!array2hash(arrayref, undef\ sub)}}
\begin{verbatim}
   Name:        array2hash() ;
   Function:    Takes an array reference and returns a hashref with value,
                index pairs.
   Args:        $_[0] = array references
                $_[1] = undefined substitution - if the cell contains an
                  undefined value, use this value as the hash key
   Returns:     $_ = hashref
                $a->{value} = row number
\end{verbatim}
\subsubsection*{replace\_undefs(arrayref, undef\_sub)\label{pibase_pm_replace_undefs_arrayref_undef_sub_}\index{pibase pm!replace\ undefs(arrayref, undef\ sub)}}
\begin{verbatim}
   Name:        array2hash() ;
   Function:    Takes an array reference and replaces (inplace) undefined
                  values to a specified substitution value.
   Args:        $_[0] = array references
                $_[1] = undefined substitution - if the cell contains an
                  undefined value, replace with this value
   Returns:     $_ = array reference
\end{verbatim}
\subsubsection*{replace\_undefs\_blanks(arrayref, undef\_sub)\label{pibase_pm_replace_undefs_blanks_arrayref_undef_sub_}\index{pibase pm!replace\ undefs\ blanks(arrayref, undef\ sub)}}
\begin{verbatim}
   Name:        array2hash() ;
   Function:    Takes an array reference and replaces (inplace) undefined and
                  blank values to a specified substitution value.
   Args:        $_[0] = array references
                $_[1] = undefined/blank substitution - if the cell
                  contains an undefined or blank value, replace with this value
   Returns:     $_ = array reference
\end{verbatim}
\subsubsection*{replace\_char(arrayref, target, replacement)\label{pibase_pm_replace_char_arrayref_target_replacement_}\index{pibase pm!replace\ char(arrayref, target, replacement)}}
\begin{verbatim}
   Name:        array2hash() ;
   Function:    Takes an array reference and replaces (inplace) target values to a specified substitution value.
   Args:        $_[0] = array reference
                $_[1] = target value
                $_[2] = substitution value
   Returns:     $_ = array reference
\end{verbatim}
\subsubsection*{mysql\_hashload(dbh, query)\label{pibase_pm_mysql_hashload_dbh_query_}\index{pibase pm!mysql\ hashload(dbh, query)}}
\begin{verbatim}
   Name:        mysql_hashload()
   Function:    Processes a 2 column query and returns a hash pointing from
                  column1 values to column2 values (use for 1:1 relationships)
   Args:        $_[0] = dbh - DBI database handle
                $_[1] = SQL SELECT command
   Returns:     $a - hashref
                $a->{col1} = col2
\end{verbatim}
\subsubsection*{mysql\_hash2load(dbh, query)\label{pibase_pm_mysql_hash2load_dbh_query_}\index{pibase pm!mysql\ hash2load(dbh, query)}}
\begin{verbatim}
   Name:        mysql_hash2load()
   Function:    Processes a 3 column query and returns a hash pointing from
                  column1 values to column2 to column3 values
                  (use for 1:1:1 relationships)
   Args:        $_[0] = dbh - DBI database handle
                $_[1] = SQL SELECT command
   Returns:     $a - hashref
                $a->{col1}->{col2} = col3 ;
\end{verbatim}
\subsubsection*{mysql\_hasharrload(dbh, query, undef\_subs)\label{pibase_pm_mysql_hasharrload_dbh_query_undef_subs_}\index{pibase pm!mysql\ hasharrload(dbh, query, undef\ subs)}}
\begin{verbatim}
   Name:        mysql_hashload()
   Function:    Processes a 2 column query and returns a hash pointing from
                  column1 values to an array of column2 values
                  (use for 1:n relationships)
   Args:        $_[0] dbh - DBI database handle
                $_[1] query - SQL format
                $_[2] undefined value substitutors - list
   Returns:     $a - hashref to arrayrefs
   $a->{col1} = [ col2_1, col2_2,... col2_n ]
\end{verbatim}
\subsubsection*{mysql\_singleval(dbh, query)\label{pibase_pm_mysql_singleval_dbh_query_}\index{pibase pm!mysql\ singleval(dbh, query)}}
\begin{verbatim}
   Function:    Processes a 1 column, 1 row query and returns a scalar
                  containing the value.
   Args:        $_[0] = dbh - DBI database handle
                $_[1] = query - SQL format
   Return:      $a - scalar
                $a = "result"
\end{verbatim}
\subsubsection*{timestamp()\label{pibase_pm_timestamp_}\index{pibase pm!timestamp()}}
\begin{verbatim}
   Function:    Returns a timestamp
   Args:        none
   Return:      $_[0] = timestamp: <4-digit YEAR><2-digit MONTH><2-digit DAY>_
                  <2-digit HOUR><2-digit MINUTE>
\end{verbatim}
\subsubsection*{get\_current\_date\_mysql()\label{pibase_pm_get_current_date_mysql_}\index{pibase pm!get\ current\ date\ mysql()}}
\begin{verbatim}
   Function:    Returns current date in YYYY-MM-DD mysql format
   Args:        none
   Return:      $_[0] = date: <4-digit YEAR>-<2-digit MONTH>-<2-digit DAY>
\end{verbatim}
\subsubsection*{timestampsec()\label{pibase_pm_timestampsec_}\index{pibase pm!timestampsec()}}
\begin{verbatim}
   Function:    Returns a second-resolution timestamp
   Args:        none
   Return:      timestamp: <4-digit YEAR><2-digit MONTH><2-digit DAY>_
                  <2-digit HOUR><2-digit MINUTE><2-digit SECOND>
\end{verbatim}
\subsubsection*{mysqlimport(file,dbspecs)\label{pibase_pm_mysqlimport_file_dbspecs_}\index{pibase pm!mysqlimport(file,dbspecs)}}
\begin{verbatim}
   Function:    load a file into a mysql database using system(mysqlimport)
   Return:      $_[0] = records imported
                $_[1] = records deleted
                $_[2] = records skipped
                $_[3] = number of warnings
\end{verbatim}
\begin{verbatim}
   Args:        $_[0] = filename
                $_[1] = database specs - hashref
\end{verbatim}
\begin{itemize}

\item db =$>$ database name
\item user =$>$ user name
\item pass =$>$ password\end{itemize}
\subsubsection*{mysql\_runcom(dbh, query, vaues)\label{pibase_pm_mysql_runcom_dbh_query_vaues_}\index{pibase pm!mysql\ runcom(dbh, query, vaues)}}
\begin{verbatim}
   Function:    Runs an non-SELECT SQL statement
   Return:      none
   Args:        $_[0] = dbh - DBI database handle
                $_[1] = SQL query command or DBI statement handle
                $_[2] = values- arrayref that hold values for '?' holders
                        in DBI statement handle
\end{verbatim}
\subsubsection*{mysql\_createtable(dbh, tablename, spec)\label{pibase_pm_mysql_createtable_dbh_tablename_spec_}\index{pibase pm!mysql\ createtable(dbh, tablename, spec)}}
\begin{verbatim}
   Function:    Creates a table (NOTE: if table already exists, drops it first).
   Return:      none
   Args:        $_[0] = dbh - DBI database handle
                $_[1] = table name
                $_[2] = spec - SQL DDL string
\end{verbatim}
\begin{verbatim}
   Example uasge:
\end{verbatim}
\begin{verbatim}
      pibase::mysql_createtable($dbh,
         $tables->{interface_contacts}->{name},
         $tables->{interface_contacts}->{spec}) ;
\end{verbatim}
\begin{verbatim}
      pibase::mysql_runcom($dbh,
         "REPLACE INTO $tables->{interface_contacts}->{meta} ".
         "( bdp_id, table_name) values($bdp_id, ".
         "\"$tables->{interface_contacts}->{name}\")") ;
\end{verbatim}
\subsubsection*{mysql\_commandline\_query(dbh, query, vaues)\label{pibase_pm_mysql_commandline_query_dbh_query_vaues_}\index{pibase pm!mysql\ commandline\ query(dbh, query, vaues)}}
\begin{verbatim}
   Function:    Runs an SQL SELECT statement on the command line,
                optionally performs a sort (on command line), and 
                displays the output to a specified file
   Return:      none
   Args:        $->{db_name} = database name
                $->{sql} = SQL query
                $->{out_fn} = file to display results to
                $->{post_sort} = optional specify field to sort
                $->{sort_order} = optional sort order (defaults ASC)
\end{verbatim}
\subsubsection*{locate\_binaries()\label{pibase_pm_locate_binaries_}\index{pibase pm!locate\ binaries()}}
\begin{verbatim}
   Function:    Returns location of binaries used in PIBASE associated activities
   Return:      $_->{program} = program location.
      perl, zcat, rigor, subset_extractor, altloc_check
   Args:        none
\end{verbatim}
\subsubsection*{safe\_move()\label{pibase_pm_safe_move_}\index{pibase pm!safe\ move()}}
\begin{verbatim}
   Function:    Safely move a file to a directory (using File::Copy::move),
                  retries 14 times, and prints an error if it didnt work
   Return:      nothing
   Args:        $_[0] = source filename
                $_[1] = target directory
\end{verbatim}
\subsubsection*{safe\_copy()\label{pibase_pm_safe_copy_}\index{pibase pm!safe\ copy()}}
\begin{verbatim}
   Function:    Safely copy a file to a directory (using File::Copy::copy),
                  retries 14 times, and prints an error if it didnt work
   Return:      nothing
   Args:        $_[0] = source filename
                $_[1] = target directory
\end{verbatim}
\clearpage
\section{pibase.pm\label{pibase_pm}\index{pibase.pm}}


Perl interface to the pibase database

\subsection*{DESCRIPTION\label{pibase_pm_DESCRIPTION}\index{pibase pm!DESCRIPTION}}


The pibase.pm perl library contains subroutines used to build and access the
PIBASE database.

\subsection*{AUTHOR\label{pibase_pm_AUTHOR}\index{pibase pm!AUTHOR}}


Fred P. Davis, HHMI-JFRC (davisf@janelia.hhmi.org)

\subsection*{SUBROUTINES\label{pibase_pm_SUBROUTINES}\index{pibase pm!SUBROUTINES}}
\subsubsection*{dbname()\label{pibase_pm_dbname_}\index{pibase pm!dbname()}}
\begin{verbatim}
   Title:       dbname()
   Function:    gets the name of the pibase database
   Args:        pibase $specs data structure
   Returns:     returns the name of the pibase mysql database
\end{verbatim}
\subsubsection*{get\_specs()\label{pibase_pm_get_specs_}\index{pibase pm!get\ specs()}}
\begin{verbatim}
   Title:       get_specs()
   Function:    gives pibase specifications
   Args:        pibase $specs data structure
   Returns:     returns completed pibase data specifications
\end{verbatim}
\subsubsection*{connect\_pibase(\$dbspecs)\label{pibase_pm_connect_pibase_dbspecs_}\index{pibase pm!connect\ pibase(\$dbspecs)}}
\begin{verbatim}
   Title:       connect_pibase()
   Function:    Connects to the pibase database.
   Args:        $_->{db}        database name
                $_->{user}      user name
                $_->{pass}      password
   Returns:     DBI database handle to pibaes
\end{verbatim}
\subsubsection*{connect\_tod(\$dbspecs)\label{pibase_pm_connect_tod_dbspecs_}\index{pibase pm!connect\ tod(\$dbspecs)}}
\begin{verbatim}
   Title:       connect_tod()
   Function:    Connects to pibase tables on disk (POD).
   Args:        $_[0] = tablename
                $_[1] = dbspecs
                $_[2] = tod_dir = location of tables on disk
\end{verbatim}
\begin{verbatim}
   Returns:     DBI database handle to table on disk
\end{verbatim}
\subsubsection*{rawselect\_tod()\label{pibase_pm_rawselect_tod_}\index{pibase pm!rawselect\ tod()}}
\begin{verbatim}
   Title:       rawselect_tod()
   Function:    performs basic SELECT statement queries on a table on disk
   Args:        $_[0] = SQL SELECT-like statement
                $_[1] = fullfile
\end{verbatim}
\subsubsection*{connect\_metatod()\label{pibase_pm_connect_metatod_}\index{pibase pm!connect\ metatod()}}
\begin{verbatim}
   Title:       connect_metatod()
   Function:    performs basic SELECT statement queries on a meta-table on disk
   Args:        $_[0] = filename
                $_[1] = tablename
                $_[2] = pibase db specs
   Returns:     dbh DBI:AnyData database handle
\end{verbatim}
\subsubsection*{rawselect\_metatod()\label{pibase_pm_rawselect_metatod_}\index{pibase pm!rawselect\ metatod()}}
\begin{verbatim}
   Title:       rawselect_metatod()
   Function:    selects specified fields from a table-on-disk.
      Note: WHERE clause does not work, this command just recognizes
            the field names and returns the appropriate columns.
   Args:        $_[0] = filename
                $_[1] = SELECT sql command
   Returns:     array of query results
\end{verbatim}
\subsubsection*{sid\_2\_domdir()\label{pibase_pm_sid_2_domdir_}\index{pibase pm!sid\ 2\ domdir()}}
\begin{verbatim}
   Title:       sid_2_domdir()
   Function:    returns the directory name where the PDB file of the
      specified domain resides.
   Args:        $_ = subset_id
   Returns:     directory name
\end{verbatim}
\subsubsection*{complete\_pibase\_specs(specs)\label{pibase_pm_complete_pibase_specs_specs_}\index{pibase pm!complete\ pibase\ specs(specs)}}
\begin{verbatim}
   Title:       complete_pibase_specs
   Function:    Fills in blanks in specs with default values.
   Input:       $_ = specs hashref
                $_->{db} = database_name
                $_->{user} = user name
                $_->{pass} = password
   Return:      specs - hashref
\end{verbatim}
\subsubsection*{load\_bdp\_ids(\$dbh, @results\_type)\label{pibase_pm_load_bdp_ids_dbh_results_type_}\index{pibase pm!load\ bdp\ ids(\$dbh, @results\ type)}}
\begin{verbatim}
   Name: load_bdp_ids()
   Function:    Returns bdp_id and depending on results_type
                specified, its relation to bdp_path and pdb_id
                in a variety of forms.
   Return:      results
   Args:        $_[0] = DBI dbh handle
                $_[1] = results type
\end{verbatim}
\begin{itemize}

\item path\_2\_bdp\_id (hash) [default]
\item bdp\_id\_2\_path (hash)
\item bdp\_id\_2\_pdb\_id (hash)
\item bdp\_id\_2\_raw\_pdb (hash)
\item pdb\_id\_2\_bdp\_id (hash)
\item bdp\_id (array)\end{itemize}
\subsubsection*{todload\_bdp\_ids(@results\_type)\label{pibase_pm_todload_bdp_ids_results_type_}\index{pibase pm!todload\ bdp\ ids(@results\ type)}}
\begin{verbatim}
   Name: todload_bdp_ids()
   Function:    Returns bdp_id and depending on results_type
                specified, its relation to bdp_path and pdb_id
                in a variety of forms.
      Analogous to load_bdp_ids() with tables-on-disk instead of DBI
   Return:      query results
   Args:        $_[0] = DBI dbh handle
                $_[1] = results type
\end{verbatim}
\begin{itemize}

\item path\_2\_bdp\_id (hash) [default]
\item bdp\_id\_2\_path (hash)
\item bdp\_id\_2\_pdb\_id (hash)
\item bdp\_id\_2\_raw\_pdb (hash)
\item pdb\_id\_2\_bdp\_id (hash)
\item bdp\_id (array)\end{itemize}
\subsubsection*{mysql\_fetchcols(dbh, query)\label{pibase_pm_mysql_fetchcols_dbh_query_}\index{pibase pm!mysql\ fetchcols(dbh, query)}}
\begin{verbatim}
   Function:    Processes an n column query and returns a list of array references, where each array reference holds all the valeus for a given column.
   Args:        $_[0] = dbh - DBI database handle
                $_[1] = SQL select query
\end{verbatim}
\begin{verbatim}
   Returns:     @_ - list of arrayref
                $a[i]->[j] ith column, jth row
\end{verbatim}
\subsubsection*{mysql\_hashindload(dbh, query)\label{pibase_pm_mysql_hashindload_dbh_query_}\index{pibase pm!mysql\ hashindload(dbh, query)}}
\begin{verbatim}
   Name:        mysql_hashindload() ;
   Function:    Processes a 1 column query and returns a hash pointing from
                  column1 values to row number.
   Args:        $_[0] = dbh - DBI database handle
                $_[1] = SQL SELECT query
                $_[2] = substitutor for undefined value
   Returns:     $_ = hashref
                $a->{col1} = row number
\end{verbatim}
\subsubsection*{array2hash(arrayref, undef\_sub)\label{pibase_pm_array2hash_arrayref_undef_sub_}\index{pibase pm!array2hash(arrayref, undef\ sub)}}
\begin{verbatim}
   Name:        array2hash() ;
   Function:    Takes an array reference and returns a hashref with value,
                index pairs.
   Args:        $_[0] = array references
                $_[1] = undefined substitution - if the cell contains an
                  undefined value, use this value as the hash key
   Returns:     $_ = hashref
                $a->{value} = row number
\end{verbatim}
\subsubsection*{replace\_undefs(arrayref, undef\_sub)\label{pibase_pm_replace_undefs_arrayref_undef_sub_}\index{pibase pm!replace\ undefs(arrayref, undef\ sub)}}
\begin{verbatim}
   Name:        array2hash() ;
   Function:    Takes an array reference and replaces (inplace) undefined
                  values to a specified substitution value.
   Args:        $_[0] = array references
                $_[1] = undefined substitution - if the cell contains an
                  undefined value, replace with this value
   Returns:     $_ = array reference
\end{verbatim}
\subsubsection*{replace\_undefs\_blanks(arrayref, undef\_sub)\label{pibase_pm_replace_undefs_blanks_arrayref_undef_sub_}\index{pibase pm!replace\ undefs\ blanks(arrayref, undef\ sub)}}
\begin{verbatim}
   Name:        array2hash() ;
   Function:    Takes an array reference and replaces (inplace) undefined and
                  blank values to a specified substitution value.
   Args:        $_[0] = array references
                $_[1] = undefined/blank substitution - if the cell
                  contains an undefined or blank value, replace with this value
   Returns:     $_ = array reference
\end{verbatim}
\subsubsection*{replace\_char(arrayref, target, replacement)\label{pibase_pm_replace_char_arrayref_target_replacement_}\index{pibase pm!replace\ char(arrayref, target, replacement)}}
\begin{verbatim}
   Name:        array2hash() ;
   Function:    Takes an array reference and replaces (inplace) target values to a specified substitution value.
   Args:        $_[0] = array reference
                $_[1] = target value
                $_[2] = substitution value
   Returns:     $_ = array reference
\end{verbatim}
\subsubsection*{mysql\_hashload(dbh, query)\label{pibase_pm_mysql_hashload_dbh_query_}\index{pibase pm!mysql\ hashload(dbh, query)}}
\begin{verbatim}
   Name:        mysql_hashload()
   Function:    Processes a 2 column query and returns a hash pointing from
                  column1 values to column2 values (use for 1:1 relationships)
   Args:        $_[0] = dbh - DBI database handle
                $_[1] = SQL SELECT command
   Returns:     $a - hashref
                $a->{col1} = col2
\end{verbatim}
\subsubsection*{mysql\_hash2load(dbh, query)\label{pibase_pm_mysql_hash2load_dbh_query_}\index{pibase pm!mysql\ hash2load(dbh, query)}}
\begin{verbatim}
   Name:        mysql_hash2load()
   Function:    Processes a 3 column query and returns a hash pointing from
                  column1 values to column2 to column3 values
                  (use for 1:1:1 relationships)
   Args:        $_[0] = dbh - DBI database handle
                $_[1] = SQL SELECT command
   Returns:     $a - hashref
                $a->{col1}->{col2} = col3 ;
\end{verbatim}
\subsubsection*{mysql\_hasharrload(dbh, query, undef\_subs)\label{pibase_pm_mysql_hasharrload_dbh_query_undef_subs_}\index{pibase pm!mysql\ hasharrload(dbh, query, undef\ subs)}}
\begin{verbatim}
   Name:        mysql_hashload()
   Function:    Processes a 2 column query and returns a hash pointing from
                  column1 values to an array of column2 values
                  (use for 1:n relationships)
   Args:        $_[0] dbh - DBI database handle
                $_[1] query - SQL format
                $_[2] undefined value substitutors - list
   Returns:     $a - hashref to arrayrefs
   $a->{col1} = [ col2_1, col2_2,... col2_n ]
\end{verbatim}
\subsubsection*{mysql\_singleval(dbh, query)\label{pibase_pm_mysql_singleval_dbh_query_}\index{pibase pm!mysql\ singleval(dbh, query)}}
\begin{verbatim}
   Function:    Processes a 1 column, 1 row query and returns a scalar
                  containing the value.
   Args:        $_[0] = dbh - DBI database handle
                $_[1] = query - SQL format
   Return:      $a - scalar
                $a = "result"
\end{verbatim}
\subsubsection*{timestamp()\label{pibase_pm_timestamp_}\index{pibase pm!timestamp()}}
\begin{verbatim}
   Function:    Returns a timestamp
   Args:        none
   Return:      $_[0] = timestamp: <4-digit YEAR><2-digit MONTH><2-digit DAY>_
                  <2-digit HOUR><2-digit MINUTE>
\end{verbatim}
\subsubsection*{get\_current\_date\_mysql()\label{pibase_pm_get_current_date_mysql_}\index{pibase pm!get\ current\ date\ mysql()}}
\begin{verbatim}
   Function:    Returns current date in YYYY-MM-DD mysql format
   Args:        none
   Return:      $_[0] = date: <4-digit YEAR>-<2-digit MONTH>-<2-digit DAY>
\end{verbatim}
\subsubsection*{timestampsec()\label{pibase_pm_timestampsec_}\index{pibase pm!timestampsec()}}
\begin{verbatim}
   Function:    Returns a second-resolution timestamp
   Args:        none
   Return:      timestamp: <4-digit YEAR><2-digit MONTH><2-digit DAY>_
                  <2-digit HOUR><2-digit MINUTE><2-digit SECOND>
\end{verbatim}
\subsubsection*{mysqlimport(file,dbspecs)\label{pibase_pm_mysqlimport_file_dbspecs_}\index{pibase pm!mysqlimport(file,dbspecs)}}
\begin{verbatim}
   Function:    load a file into a mysql database using system(mysqlimport)
   Return:      $_[0] = records imported
                $_[1] = records deleted
                $_[2] = records skipped
                $_[3] = number of warnings
\end{verbatim}
\begin{verbatim}
   Args:        $_[0] = filename
                $_[1] = database specs - hashref
\end{verbatim}
\begin{itemize}

\item db =$>$ database name
\item user =$>$ user name
\item pass =$>$ password\end{itemize}
\subsubsection*{mysql\_runcom(dbh, query, vaues)\label{pibase_pm_mysql_runcom_dbh_query_vaues_}\index{pibase pm!mysql\ runcom(dbh, query, vaues)}}
\begin{verbatim}
   Function:    Runs an non-SELECT SQL statement
   Return:      none
   Args:        $_[0] = dbh - DBI database handle
                $_[1] = SQL query command or DBI statement handle
                $_[2] = values- arrayref that hold values for '?' holders
                        in DBI statement handle
\end{verbatim}
\subsubsection*{mysql\_createtable(dbh, tablename, spec)\label{pibase_pm_mysql_createtable_dbh_tablename_spec_}\index{pibase pm!mysql\ createtable(dbh, tablename, spec)}}
\begin{verbatim}
   Function:    Creates a table (NOTE: if table already exists, drops it first).
   Return:      none
   Args:        $_[0] = dbh - DBI database handle
                $_[1] = table name
                $_[2] = spec - SQL DDL string
\end{verbatim}
\begin{verbatim}
   Example uasge:
\end{verbatim}
\begin{verbatim}
      pibase::mysql_createtable($dbh,
         $tables->{interface_contacts}->{name},
         $tables->{interface_contacts}->{spec}) ;
\end{verbatim}
\begin{verbatim}
      pibase::mysql_runcom($dbh,
         "REPLACE INTO $tables->{interface_contacts}->{meta} ".
         "( bdp_id, table_name) values($bdp_id, ".
         "\"$tables->{interface_contacts}->{name}\")") ;
\end{verbatim}
\subsubsection*{mysql\_commandline\_query(dbh, query, vaues)\label{pibase_pm_mysql_commandline_query_dbh_query_vaues_}\index{pibase pm!mysql\ commandline\ query(dbh, query, vaues)}}
\begin{verbatim}
   Function:    Runs an SQL SELECT statement on the command line,
                optionally performs a sort (on command line), and 
                displays the output to a specified file
   Return:      none
   Args:        $->{db_name} = database name
                $->{sql} = SQL query
                $->{out_fn} = file to display results to
                $->{post_sort} = optional specify field to sort
                $->{sort_order} = optional sort order (defaults ASC)
\end{verbatim}
\subsubsection*{locate\_binaries()\label{pibase_pm_locate_binaries_}\index{pibase pm!locate\ binaries()}}
\begin{verbatim}
   Function:    Returns location of binaries used in PIBASE associated activities
   Return:      $_->{program} = program location.
      perl, zcat, rigor, subset_extractor, altloc_check
   Args:        none
\end{verbatim}
\subsubsection*{safe\_move()\label{pibase_pm_safe_move_}\index{pibase pm!safe\ move()}}
\begin{verbatim}
   Function:    Safely move a file to a directory (using File::Copy::move),
                  retries 14 times, and prints an error if it didnt work
   Return:      nothing
   Args:        $_[0] = source filename
                $_[1] = target directory
\end{verbatim}
\subsubsection*{safe\_copy()\label{pibase_pm_safe_copy_}\index{pibase pm!safe\ copy()}}
\begin{verbatim}
   Function:    Safely copy a file to a directory (using File::Copy::copy),
                  retries 14 times, and prints an error if it didnt work
   Return:      nothing
   Args:        $_[0] = source filename
                $_[1] = target directory
\end{verbatim}
\clearpage
\section{pibase::ASTRAL\label{pibase::ASTRAL}\index{pibase::ASTRAL}}


Perl module to access ASTRAL data

\subsection*{DESCRIPTION\label{pibase::ASTRAL_DESCRIPTION}\index{pibase::ASTRAL!DESCRIPTION}}


Perl module that contains routines for accessing ASTRAL data, for
use in clustering PIBASE interfaces

\subsection*{AUTHOR\label{pibase::ASTRAL_AUTHOR}\index{pibase::ASTRAL!AUTHOR}}


Fred P. Davis, HHMI-JFRC (davisf@janelia.hhmi.org)

\subsection*{SUBROUTINES\label{pibase::ASTRAL_SUBROUTINES}\index{pibase::ASTRAL!SUBROUTINES}}
\subsubsection*{load\_asteroids\_aln()\label{pibase::ASTRAL_load_asteroids_aln_}\index{pibase::ASTRAL!load\ asteroids\ aln()}}
\begin{verbatim}
   Title:       load_asteroids_aln()
   Function:    Loads an ASTRAL ASTEROIDS alignment file
   Args:        $_->{aln_fn}    name of ASTERIODS alignment file
                $_->{seq_fn}    name of corresponding ASTEROIDS sequence file
                $_->{allchains} 
                $_->{gdseqh}    data structure holding contents of gdseqh file
                $_->{seqclcont100}
                $_->{seqcl100}
                $_->{doms}      optional hash list of domains to load.
   Returns:     parse_aln_raf() alignment structure
\end{verbatim}
\subsubsection*{raf\_preload()\label{pibase::ASTRAL_raf_preload_}\index{pibase::ASTRAL!raf\ preload()}}
\begin{verbatim}
   Title:       raf_preload()
   Function:    Loads the ASTRAL raf file
   Args:        $_->{fn}        name of the ASTRAL raf file
   Returns:     ->{pdb_chain} = "RAF line contents"
\end{verbatim}
\subsubsection*{parse\_raf\_line()\label{pibase::ASTRAL_parse_raf_line_}\index{pibase::ASTRAL!parse\ raf\ line()}}
\begin{verbatim}
   Title:       parse_raf_line()
   Function:    Parses a line from ASTRAL raf file
   Args:        $_->{line}      RAF line
                $_->{headlength}        length of RAF file header
\end{verbatim}
\begin{verbatim}
   Returns:     ->{atomresno_first}     ATOM residue number of first residue
                ->{atomresno_last}      ATOM residue number of last residue
                ->{atomresna}   = [ ATOM residue name 1,2, ... ]
                ->{atomresno}   = [ ATOM residue number 1,2, ... ]
                ->{seqresna}    = [ sequence residue name 1,2, ... ]
                ->{seqresno}    = [ sequence residue number 1,2, ... ]
                ->{seqresno2ind}->{seqresno} = index in @{->{seqresno}}
                ->{ind2seqresno}->{index in @{->{seqresno}}} = seqresno
                ->{atom2seqresno_back}->{$atomresno} = seqresno
                ->{atomresno2ind_back}->{$atomresno} = seqresno index
                ->{seq2atomresno}->{$seqresno} = $atomresno ;
                ->{atom2seqresno}->{$atomresno} = $seqresno ;
                ->{atomresno2ind}->{$atomresno} = $#{$res->{seqresno}} ;
                ->{ind2atomresno}->{$#{$res->{seqresno}}} = $atomresno  ;
\end{verbatim}
\subsubsection*{read\_asteroids\_aln()\label{pibase::ASTRAL_read_asteroids_aln_}\index{pibase::ASTRAL!read\ asteroids\ aln()}}
\begin{verbatim}
   Title:       read_asteroids_aln()
   Function:    Reads an ASTEROIDS alignment file
   Args:        $_->{aln_fn} ASTEROIDS alignment file name
                $_->{seq_fn} corresponding ASTEROIDS sequence file name
                $_->{allchains}->{domain} = pdb chain
\end{verbatim}
\begin{verbatim}
   Returns:     ->{seq}->{domain} = 'DOMAINSEQVENCE';
                ->{defstring}->{domain} = definition line from alignment
                ->{class}->{domain} = SCOP class
                ->{aln}->{domain} = domain sequence froma alignment
                ->{pdb}->{domain} = PDB code for the domain
                ->{frags}->{domain} = [{b => startresidue, e => endresidue},...]
                ->{alnlength} = alignment length
\end{verbatim}
\subsubsection*{parse\_aln\_raf()\label{pibase::ASTRAL_parse_aln_raf_}\index{pibase::ASTRAL!parse\ aln\ raf()}}
\begin{verbatim}
   Title:       parse_aln_raf()
   Function:    Reads an ASTEROIDS alignment file
   Args:        $_->{alndata}  - alignment data from read_asteroids_aln()
                $_->{raf} - RAF data from raf_preload()
\end{verbatim}
\begin{verbatim}
   Returns:     ->{pos2resno}->{domain}->{alignment position} = ATOM resno
                ->{resno2pos}->{domain}->{ATOM resno} = alignment position
\end{verbatim}
\subsubsection*{load\_astral\_headers()\label{pibase::ASTRAL_load_astral_headers_}\index{pibase::ASTRAL!load\ astral\ headers()}}
\begin{verbatim}
   Title:       load_astral_headers()
   Function:    Loads ASTRAL headers
   Args:        $_->{fn} - alignment file name
                $_->{raf} - RAF data from raf_preload()
\end{verbatim}
\begin{verbatim}
   Returns:     ->{gdseqh}->{defstring}->{domain} = domain definition string
                ->{gdseqh}->{class}->{domain} = domain class
                ->{gdseqh}->{pdb}->{domain} = domain PDB code
                ->{gdseqh}->{frags}->{domain} = [{b => startres, e => endres}...]
                ->{gdom}->{domain (w d prefix)} = domain (w g prefix)
\end{verbatim}
\subsubsection*{load\_astral\_clusters()\label{pibase::ASTRAL_load_astral_clusters_}\index{pibase::ASTRAL!load\ astral\ clusters()}}
\begin{verbatim}
   Title:       load_astral_clusters()
   Function:    Loads ASTRAL sequence cluster definitions
   Args:        $_->{out} - pointer to hash to hold output
                $_->{pibase_specs} -  pibase_specs structure
\end{verbatim}
\begin{verbatim}
   Returns:     Nothing - populates the specified $_->{out}
                {out}->{seqcl}->{seq identity}->{scop identifier} = cluster num
                {out}->{seqcl2cont}->{seq identity}->{cluster num} = [scop id,...]
\end{verbatim}
\subsubsection*{get\_astral\_classlist()\label{pibase::ASTRAL_get_astral_classlist_}\index{pibase::ASTRAL!get\ astral\ classlist()}}
\begin{verbatim}
   Title:       get_astral_classlist()
   Function:    Get list of SCOP classes in the ASTRAL compendium
   Args:        $_->{pibase_specs} -  pibase_specs structure
   Returns:     Nothing - populates the specified $_->{out}
                ->{fam}->{scop_family} = number of domains in the family
                ->{sf}->{scop_superfamily} = number of domains in the superfamily
\end{verbatim}
\clearpage
\section{pibase::CATH\label{pibase::CATH}\index{pibase::CATH}}


Interface for CATH domain database data processing

\subsection*{DESCRIPTION\label{pibase::CATH_DESCRIPTION}\index{pibase::CATH!DESCRIPTION}}


This module contains routines to process and reformat CATH release files
for PIBASE import.  Files are from: http://www.cathdb.info

\subsection*{AUTHOR\label{pibase::CATH_AUTHOR}\index{pibase::CATH!AUTHOR}}


Fred P. Davis, HHMI-JFRC (davisf@janelia.hhmi.org)

\subsection*{SUBROUTINES\label{pibase::CATH_SUBROUTINES}\index{pibase::CATH!SUBROUTINES}}
\subsubsection*{cath\_clean\_cddf()\label{pibase::CATH_cath_clean_cddf_}\index{pibase::CATH!cath\ clean\ cddf()}}
\begin{verbatim}
   Title:       cath_clean_cddf()
   Function:    cleans the CATH CDDF file
   STDIN:       CATH CDDF file
   STDOUT:      cleaned up CATH CDDF file
   Args:        nothing
   Returns:     nothing
\end{verbatim}
\subsubsection*{cath\_parse\_cdf\_domainlist()\label{pibase::CATH_cath_parse_cdf_domainlist_}\index{pibase::CATH!cath\ parse\ cdf\ domainlist()}}
\begin{verbatim}
   Title:       cath_parse_cdf_domainlist()
   Function:    cleans the CATH CDDF file
   STDIN:       CATH CDDF file
   STDOUT:      cleaned up CATH CDDF file
   Args:        nothing
   Returns:     nothing
\end{verbatim}
\subsubsection*{cath\_clean\_clf()\label{pibase::CATH_cath_clean_clf_}\index{pibase::CATH!cath\ clean\ clf()}}
\begin{verbatim}
   Title:       cath_clean_clf()
   Function:    cleans the CATH CLF file
   STDIN:       CATH CLF file
   STDOUT:      cleaned up CATH CLF file
   Args:        nothing
   Returns:     nothing
\end{verbatim}
\subsubsection*{cath\_clean\_cnf\_contraction()\label{pibase::CATH_cath_clean_cnf_contraction_}\index{pibase::CATH!cath\ clean\ cnf\ contraction()}}
\begin{verbatim}
   Title:       cath_clean_cnf_contraction()
   Function:    fixes the concatenation problem with the CATH CNF file
   STDIN:       CATH CNF file
   STDOUT:      concat-fixed CATH CNF file
   Args:        nothing
   Returns:     nothing
\end{verbatim}
\subsubsection*{cath\_clean\_cnf\_contraction()\label{pibase::CATH_cath_clean_cnf_contraction_}\index{pibase::CATH!cath\ clean\ cnf\ contraction()}}
\begin{verbatim}
   Title:       cath_clean_cnf()
   Function:    fixes the concatenation problem with the CATH CNF file
   STDIN:       CATH CNF file (preferably concat-fixed)
   STDOUT:      cleaned CATH CNF file
   Args:        nothing
   Returns:     nothing
\end{verbatim}
\subsubsection*{pibase\_import\_cath\_domains()\label{pibase::CATH_pibase_import_cath_domains_}\index{pibase::CATH!pibase\ import\ cath\ domains()}}
\begin{verbatim}
   Title:       pibase_import_cath_domains()
   Function:    reformats raw CATH tables imported into pibase as generic
                  subsets tables
   In tables:   1. cath_domain_list
                2. cath_domall_boundaries
                3. cath_names
\end{verbatim}
\begin{verbatim}
   Out tables:  1. subsets
                2. subsets_class
                3. subsets_details
\end{verbatim}
\begin{verbatim}
   Args:        nothing
   Returns:     nothing
\end{verbatim}
\clearpage
\section{pibase::PDB\label{pibase::PDB}\index{pibase::PDB}}


Module to handle PDB functions

\subsection*{DESCRIPTION\label{pibase::PDB_DESCRIPTION}\index{pibase::PDB!DESCRIPTION}}


Handles general PDB functions: altloc check, filter

\subsection*{FILES\label{pibase::PDB_FILES}\index{pibase::PDB!FILES}}


Operates on PDB file

\subsection*{AUTHOR\label{pibase::PDB_AUTHOR}\index{pibase::PDB!AUTHOR}}


Fred P. Davis, HHMI-JFRC (davisf@janelia.hhmi.org)

\subsection*{SUBROUTINES\label{pibase::PDB_SUBROUTINES}\index{pibase::PDB!SUBROUTINES}}
\subsubsection*{altloc\_check()\label{pibase::PDB_altloc_check_}\index{pibase::PDB!altloc\ check()}}
\begin{verbatim}
   Title:       altloc_check()
   Function:    checks whether a PDB file for atoms with multiple locations
   Args:        $_ = pdb filename
   Return:      1 if containts multiple-occurrence atoms, 0 if not
\end{verbatim}
\subsubsection*{altloc\_filter()\label{pibase::PDB_altloc_filter_}\index{pibase::PDB!altloc\ filter()}}
\begin{verbatim}
   Title:       altloc_filter()
   Function:    calls the altloc_filter binary so that each atom occurs once
      Leaves the highest occupied (if occupancy defined) or the first location
      listed in the file.
\end{verbatim}
\begin{verbatim}
   Args:        $_[0] = source pdb_filename
                $_[1] = output pdb_filename
\end{verbatim}
\begin{verbatim}
   Returns:     nothing
\end{verbatim}
\subsubsection*{pdb\_copy\_entry\_type()\label{pibase::PDB_pdb_copy_entry_type_}\index{pibase::PDB!pdb\ copy\ entry\ type()}}
\begin{verbatim}
   Title:       pdb_copy_entry_type()
   Function:    copies PDB pdb_entry_type for pibase import
   STDIN:       PDB pdb_entry_type
   STDOUT:      pdb_entry_type.pibase_id pibase table
   Returns:     nothing
\end{verbatim}
\subsubsection*{pdb\_clean\_entries\_idx()\label{pibase::PDB_pdb_clean_entries_idx_}\index{pibase::PDB!pdb\ clean\ entries\ idx()}}
\begin{verbatim}
   Title:       pdb_clean_entries_idx()
   Function:    reformats the PDB entries.idx for pibase import
   STDIN:       PDB entries.idx
   STDOUT:      pibase.pdb_entries table
   Returns:     nothing
\end{verbatim}
\subsubsection*{pdb\_clean\_obsolete\_dat()\label{pibase::PDB_pdb_clean_obsolete_dat_}\index{pibase::PDB!pdb\ clean\ obsolete\ dat()}}
\begin{verbatim}
   Title:       pdb_clean_obsolete_dat()
   Function:    reformats the PDB obsolete.dat for pibase import
   STDIN:       PDB obsolete.dat
   STDOUT:      pibase.pdb_obsolete table
   Returns:     nothing
\end{verbatim}
\subsubsection*{pdb\_clean\_release\_date()\label{pibase::PDB_pdb_clean_release_date_}\index{pibase::PDB!pdb\ clean\ release\ date()}}
\begin{verbatim}
   Title:       pdb_clean_release_date()
   Function:    reformats the PDB release file for pibase import
   STDIN:       NOTDONE PDB obsolete.dat
   STDOUT:      pibase.pdb_release table
   Returns:     nothing
\end{verbatim}
\subsubsection*{pdb\_clean\_symop()\label{pibase::PDB_pdb_clean_symop_}\index{pibase::PDB!pdb\ clean\ symop()}}
\begin{verbatim}
   Title:       pdb_clean_symop()
   Function:    Extracts and displays symmetry operators:
   STDIN:       PDB symop lines
   STDOUT:      pibase.pdb_release table
   Returns:     nothing
\end{verbatim}
\subsubsection*{get\_pdb\_filepath()\label{pibase::PDB_get_pdb_filepath_}\index{pibase::PDB!get\ pdb\ filepath()}}
\begin{verbatim}
   Title:       get_pdb_filepath()
   Function:    returns file path to a PDB entry
   Args:        ->{pdb_id} = pdb identifier
                [->{pibase_specs} = $pibase_specs] - optional
   Returns:     returns PDB entry filepath
\end{verbatim}
\subsubsection*{nmr\_model1\_extractor\label{pibase::PDB_nmr_model1_extractor}\index{pibase::PDB!nmr\ model1\ extractor}}
\begin{verbatim}
   Title:       nmr_model1_extractor()
   Function:    extracts first model from NMR PDB files and moves to 
                $specs->{pdbnmr_dir}
   Args:        none
   Returns:     nothing
   Tables in:   pibase.pdb_entries
   Files in:    foreach PDB NMR entry: <$specs->{pdb_dir}>/pdb<$pdb_id>.ent
   Files out:   foreach PDB NMR entry: <$specs->{pdbnmr_dir}>/<$pdb_id>_1.ent
   NOTE: expects PDBs to be stored as pdb<$pdb_id>.ent in one raw directory
\end{verbatim}
\clearpage
\section{pibase::PDB::chains\label{pibase::PDB::chains}\index{pibase::PDB::chains}}


Perl module to extract chain listing from a pdb file.

\subsection*{DESCRIPTION\label{pibase::PDB::chains_DESCRIPTION}\index{pibase::PDB::chains!DESCRIPTION}}


Perl module to parse a pdb file and lists the chains.

\subsection*{AUTHOR\label{pibase::PDB::chains_AUTHOR}\index{pibase::PDB::chains!AUTHOR}}


Fred P. Davis, HHMI-JFRC (davisf@janelia.hhmi.org)

\subsection*{SUBROUTINES\label{pibase::PDB::chains_SUBROUTINES}\index{pibase::PDB::chains!SUBROUTINES}}
\subsubsection*{SUB chain\_info()\label{pibase::PDB::chains_SUB_chain_info_}\index{pibase::PDB::chains!SUB chain\ info()}}
\begin{verbatim}
   Title: chain_info()
   Function: calculates chain listing for a pdb file
\end{verbatim}
\begin{verbatim}
   Args:        $_[0] = pdb_file
                $_[1] = outfile
                $_[2] = bdp_id identifier [optional]
   Returns:     nothing
\end{verbatim}
\begin{verbatim}
   Input file:  PDB file ($_[0]) 
      http://www.rcsb.org/pdb/docs/format/pdbguide2.2/guide2.2_frame.html
      http://msdlocal.ebi.ac.uk/docs/pdb_format/y_index.html
\end{verbatim}
\begin{verbatim}
   Output file: Chain listing ($_[1])
      1 pdb identifier (e.g. bdp_id)
      2 Chain number
      3 Chain id
      4 Chain type - p (protein) or n (nucleic acid)
      5 null
      6 null
      7 start residue number
      8 start residue number (integer only)
      9 end residue number
     10 end residue number (integer only)
     11 number of residues
     12 number of atoms
     13 number of het atoms
     14 chain sequence
\end{verbatim}
\clearpage
\section{pibase::PDB::residues\label{pibase::PDB::residues}\index{pibase::PDB::residues}}


Perl package to extract residue listing from a PDB file.

\subsection*{DESCRIPTION\label{pibase::PDB::residues_DESCRIPTION}\index{pibase::PDB::residues!DESCRIPTION}}


Parses a pdb file and lists the residues.

\subsection*{AUTHOR\label{pibase::PDB::residues_AUTHOR}\index{pibase::PDB::residues!AUTHOR}}


Fred P. Davis, HHMI-JFRC (davisf@janelia.hhmi.org)

\clearpage
\section{pibase::PDB::sec\_strx\label{pibase::PDB::sec_strx}\index{pibase::PDB::sec\ strx}}


Obtain secondary structure assignments from dsspcmbi.

\subsection*{DESCRIPTION\label{pibase::PDB::sec_strx_DESCRIPTION}\index{pibase::PDB::sec strx!DESCRIPTION}}


Interface to DSSP to calculate secondary structure for PIBASE structures.

\subsection*{AUTHOR\label{pibase::PDB::sec_strx_AUTHOR}\index{pibase::PDB::sec strx!AUTHOR}}


Fred P. Davis, HHMI-JFRC (davisf@janelia.hhmi.org)

\subsection*{SUBROUTINES\label{pibase::PDB::sec_strx_SUBROUTINES}\index{pibase::PDB::sec strx!SUBROUTINES}}
\subsubsection*{SUB get\_sec\_strx()\label{pibase::PDB::sec_strx_SUB_get_sec_strx_}\index{pibase::PDB::sec strx!SUB get\ sec\ strx()}}
\begin{verbatim}
   Function: calls DSSP to get a PDB file's residue secondary structure assignment
   Args: $_[0] - pdb file
         $_[1] - output file
         $_[2] - bdp identifier - e.g. bdp_id [optional]
\end{verbatim}
\begin{verbatim}
   Returns:     nothing
   Output file:
      1 pdb identifier (e.g. bdp_id)
      2 Chain number
      3 Chain id
      4 residue number
      5 residue number (integer only)
      6 residue name
      7 Polymer type - p (protein) or n (nucleic acid)
\end{verbatim}
\subsubsection*{SUB parse\_dssp()\label{pibase::PDB::sec_strx_SUB_parse_dssp_}\index{pibase::PDB::sec strx!SUB parse\ dssp()}}
\begin{verbatim}
   Title:    parse_dssp()
   Function: parses DSSP output and returns a hash of assignment data
   Args: $_[0] - DSSP output file
         $_[1] - output file
         $_[2] - bdp identifier - e.g. bdp_id [optional]
\end{verbatim}
\begin{verbatim}
   Returns:     $->{detail}->{resno."\n".chain_id} = H|G|I|B|E|T|S|' '
                $->{basic}->{resno."\n".chain_id} = H|B|T|' '
                $->{ordering} = [resno1."\n".chain_id1, resno2."\n".chain_id2...]
                $->{ssnum}->{resno."\n".chain_id} = secondary structure element #
                   counts contiguous stretches of same particular detailed sec
                   strx assignment)
                $->{ssnum_basic}->{resno."\n".chain_id} = secondary structure
                   element number - counts contiguous stretches of basic secondary
                   structurea ssignment
\end{verbatim}
\clearpage
\section{pibase::subsets\label{pibase::subsets}\index{pibase::subsets}}


Perl module that deals with PDB subsets

\subsection*{DESCRIPTION\label{pibase::subsets_DESCRIPTION}\index{pibase::subsets!DESCRIPTION}}


Perl module to handle subset operations on PDB files

\subsection*{AUTHOR\label{pibase::subsets_AUTHOR}\index{pibase::subsets!AUTHOR}}


Fred P. Davis, HHMI-JFRC (davisf@janelia.hhmi.org)

\subsection*{SUBROUTINES\label{pibase::subsets_SUBROUTINES}\index{pibase::subsets!SUBROUTINES}}
\subsubsection*{SUB subset\_extract()\label{pibase::subsets_SUB_subset_extract_}\index{pibase::subsets!SUB subset\ extract()}}
\begin{verbatim}
   Function: extracts specified residues/chains from PDB file
   Args:        $_[0] = PDB file name
                $_[1] = output PDB file name
                $_[2] = chain identifier
                $_[3] = start residue number
                $_[4] = end residue number
\end{verbatim}
\begin{verbatim}
   Returns:     $_->[i] arrayref of errors
\end{verbatim}
\begin{verbatim}
   Files IN:    PDB file ($_[0])
   Files OUT:   subset PDB file ($_[1])
\end{verbatim}
\clearpage
\section{pibase::PISA\label{pibase::PISA}\index{pibase::PISA}}


Perl module of routines that operate on PISA (PDBe's Protein
Interfaces, Surfaces, and Assemblies) files.

\subsection*{DESCRIPTION\label{pibase::PISA_DESCRIPTION}\index{pibase::PISA!DESCRIPTION}}


The PISA.pm module contains routines to process and format PISA release files for PIBASE import.

\subsection*{AUTHOR\label{pibase::PISA_AUTHOR}\index{pibase::PISA!AUTHOR}}


Fred P. Davis, HHMI-JFRC (davisf@janelia.hhmi.org)

\subsection*{SUBROUTINES\label{pibase::PISA_SUBROUTINES}\index{pibase::PISA!SUBROUTINES}}
\subsubsection*{pisa\_clean\_index()\label{pibase::PISA_pisa_clean_index_}\index{pibase::PISA!pisa\ clean\ index()}}
\begin{verbatim}
   Title:       pisa_clean_index()
   Function:    cleans PISA index.txt file for pibase import
   Args:        none
   STDIN:       INDEX
   STDOUT:      pibase.pisa_index table
\end{verbatim}
\subsubsection*{get\_pisa\_filepath()\label{pibase::PISA_get_pisa_filepath_}\index{pibase::PISA!get\ pisa\ filepath()}}
\begin{verbatim}
   Title:       get_pisa_filepath()
   Function:    returns file path to a PISA entry
   Args:        ->{pisa_id} = PISA identifier
                [->{pibase_specs} = $pibase_specs] - optional
   Returns:     returns PISA entry filepath
\end{verbatim}
\clearpage
\section{pibase::PQS\label{pibase::PQS}\index{pibase::PQS}}


Perl module of routines that operate on PQS (EBI's Probable
Quaternary Structure server) release files.

\subsection*{DESCRIPTION\label{pibase::PQS_DESCRIPTION}\index{pibase::PQS!DESCRIPTION}}


The PQS.pm module contains routines to process and format PQS release files for PIBASE import.

\subsection*{AUTHOR\label{pibase::PQS_AUTHOR}\index{pibase::PQS!AUTHOR}}


Fred P. Davis, HHMI-JFRC (davisf@janelia.hhmi.org)

\subsection*{SUBROUTINES\label{pibase::PQS_SUBROUTINES}\index{pibase::PQS!SUBROUTINES}}
\subsubsection*{pqs\_clean\_asalist()\label{pibase::PQS_pqs_clean_asalist_}\index{pibase::PQS!pqs\ clean\ asalist()}}
\begin{verbatim}
   Title:       pqs_clean_asalist()
   Function:    cleans PQS ASALIST file for pibase import
   Args:        none
   STDIN:       ASALIST
   STDOUT:      pibase.pqs_asalist table
\end{verbatim}
\subsubsection*{pqs\_clean\_biolist()\label{pibase::PQS_pqs_clean_biolist_}\index{pibase::PQS!pqs\ clean\ biolist()}}
\begin{verbatim}
   Title:       pqs_clean_biolist()
   Function:    cleans PQS BIOLIST file for pibase import
   Args:        none
   STDIN:       BIOLIST (preferably pqs_clean_biolist_contract() fixed)
   STDOUT:      pibase.pqs_biolist table
\end{verbatim}
\subsubsection*{pqs\_clean\_biolist\_contraction()\label{pibase::PQS_pqs_clean_biolist_contraction_}\index{pibase::PQS!pqs\ clean\ biolist\ contraction()}}
\begin{verbatim}
   Title:       pqs_clean_biolist_contraction()
   Function:    fixes contraction errors in PQS BIOLIST
   Args:        none
   STDIN:       BIOLIST
   STDOUT:      contraction fixed BIOLIST
\end{verbatim}
\subsubsection*{pqs\_clean\_list()\label{pibase::PQS_pqs_clean_list_}\index{pibase::PQS!pqs\ clean\ list()}}
\begin{verbatim}
   Title:       pqs_clean_list()
   Function:    reformats PQS LIST for pibase import
   Args:        none
   STDIN:       PQS LIST
   STDOUT:      pibase.pqs_list table
\end{verbatim}
\subsubsection*{pqs\_clean\_ranking()\label{pibase::PQS_pqs_clean_ranking_}\index{pibase::PQS!pqs\ clean\ ranking()}}
\begin{verbatim}
   Title:       pqs_clean_ranking()
   Function:    reformats PQS RANKING for pibase import
   Args:        none
   STDIN:       PQS RANKING
   STDOUT:      pibase.pqs_ranking table
\end{verbatim}
\subsubsection*{get\_pqs\_filepath()\label{pibase::PQS_get_pqs_filepath_}\index{pibase::PQS!get\ pqs\ filepath()}}
\begin{verbatim}
   Title:       get_pqs_filepath()
   Function:    returns file path to a PQS entry
   Args:        ->{pqs_id} = PQS identifier
                [->{pibase_specs} = $pibase_specs] - optional
   Returns:     returns PQS entry filepath
\end{verbatim}
\clearpage
\section{pibase::SCOP\label{pibase::SCOP}\index{pibase::SCOP}}


Package that handles SCOP release files and pibase.

\subsection*{DESCRIPTION\label{pibase::SCOP_DESCRIPTION}\index{pibase::SCOP!DESCRIPTION}}


Processes SCOP release files and

\subsection*{AUTHOR\label{pibase::SCOP_AUTHOR}\index{pibase::SCOP!AUTHOR}}


Fred P. Davis, HHMI-JFRC (davisf@janelia.hhmi.org)

\subsection*{SUBROUTINES\label{pibase::SCOP_SUBROUTINES}\index{pibase::SCOP!SUBROUTINES}}
\subsubsection*{scop\_clean\_cla()\label{pibase::SCOP_scop_clean_cla_}\index{pibase::SCOP!scop\ clean\ cla()}}
\begin{verbatim}
   Title:       scop_clean_cla()
   Function:    Processes the SCOP .cla file for pibase import
   STDIN:       SCOP CLA file
   STDOUT:      pibase.scop_cla table
   Args:        none
   Returns:     none
\end{verbatim}
\subsubsection*{scop\_clean\_des()\label{pibase::SCOP_scop_clean_des_}\index{pibase::SCOP!scop\ clean\ des()}}
\begin{verbatim}
   Title:       scop_clean_des()
   Function:    Processes the SCOP .des file for pibase import
   STDIN:       SCOP DES file
   STDOUT:      pibase.scop_des table
   Args:        none
   Returns:     none
\end{verbatim}
\subsubsection*{scop\_clean\_hie()\label{pibase::SCOP_scop_clean_hie_}\index{pibase::SCOP!scop\ clean\ hie()}}
\begin{verbatim}
   Title:       scop_clean_hie()
   Function:    Processes the SCOP .hie file for pibase import
   STDIN:       SCOP HIE file
   STDOUT:      pibase.scop_hie table
   Args:        $_->{in_fn} = input file
                $_->{out_fn} = output file
                $_->{header_fl} = flag to generate header in output (default 1)
   Returns:     none
\end{verbatim}
\subsubsection*{pibase\_import\_scop\_domains()\label{pibase::SCOP_pibase_import_scop_domains_}\index{pibase::SCOP!pibase\ import\ scop\ domains()}}
\begin{verbatim}
   Title:       pibase_import_scop_domains()
   Function:    Processes the SCOP .hie file for pibase import
   Tables in:   pibase.scop_cla
                pibase.scop_des
\end{verbatim}
\begin{verbatim}
   Tables out:  pibase.subsets
                pibase.subsets_class
                pibase.subsets_details
   Args:        none
   Returns:     none
\end{verbatim}
\clearpage
\section{pibase::SGE\label{pibase::SGE}\index{pibase::SGE}}


Perl module for SGE cluster interaction

\subsection*{DESCRIPTION\label{pibase::SGE_DESCRIPTION}\index{pibase::SGE!DESCRIPTION}}


Perl module with routines to interact with an SGE cluster,
adapted from routines in modtie.pm

\subsection*{AUTHOR\label{pibase::SGE_AUTHOR}\index{pibase::SGE!AUTHOR}}


Fred P. Davis, HHMI-JFRC (davisf@janelia.hhmi.org)

\clearpage
\section{pibase::SUPFAM.pm\label{pibase::SUPFAM_pm}\index{pibase::SUPFAM.pm}}




\subsection*{DESCRIPTION\label{pibase::SUPFAM_pm_DESCRIPTION}\index{pibase::SUPFAM pm!DESCRIPTION}}


This module contains routines to interface with SUPFAM annotation files.
Goal is to map SCOP residue numbers onto target sequences using
SUPFAM alignments and ASTRAL mapping.

\subsection*{VERSION\label{pibase::SUPFAM_pm_VERSION}\index{pibase::SUPFAM pm!VERSION}}


fpd091013\_0708

\subsection*{AUTHOR\label{pibase::SUPFAM_pm_AUTHOR}\index{pibase::SUPFAM pm!AUTHOR}}


Fred P. Davis, HHMI-JFRC (davisf@janelia.hhmi.org)

\subsection*{SUBROUTINES\label{pibase::SUPFAM_pm_SUBROUTINES}\index{pibase::SUPFAM pm!SUBROUTINES}}
\subsubsection*{set\_supfam\_specs\label{pibase::SUPFAM_pm_set_supfam_specs}\index{pibase::SUPFAM pm!set\ supfam\ specs}}
\begin{verbatim}
   Title:       set_supfam_specs()
   Function:    Sets configuration parameters
   Args:        None
   Returns:     $_->{option} = value; hash of parameters
\end{verbatim}
\subsubsection*{readin\_substitution\_matrix\label{pibase::SUPFAM_pm_readin_substitution_matrix}\index{pibase::SUPFAM pm!readin\ substitution\ matrix}}
\begin{verbatim}
   Title:       readin_substitution_matrix()
   Function:    Parses a substitution matrix in matblas format
   Args:        ->{matrix_fn} = filename of substitution matrix
                ->{string} = string containing contents of matrix file
\end{verbatim}
\begin{verbatim}
   Returns:     ->{raw}->{aa1}->{aa2} = substitution matrix score for aa1,aa2
                ->{nl}->{aa1}->{aa2} = normalized aa similarity score
                  (see karling_normalize_matrix() for normalization scheme)
\end{verbatim}
\subsubsection*{karlin\_normalize\_matrix\label{pibase::SUPFAM_pm_karlin_normalize_matrix}\index{pibase::SUPFAM pm!karlin\ normalize\ matrix}}
\begin{verbatim}
   Title:       karlin_normalize_matrix()
   Function:    Normalizes a substitution matrix per Karlin and Brocchieri,
                  J Bacteriol 1996; and rescaled to range from 0-1:
\end{verbatim}
\begin{verbatim}
               0.5 * (1 + ( mat(i,j) / sqrt(|mat(i,i) * mat(j,j)|)))
\end{verbatim}
\begin{verbatim}
   Args:        ->{matrix}->{aa1}->{aa2} = raw substitution matrix
   Returns:     ->{aa1}->{aa2} = normalized similarity score
\end{verbatim}
\subsubsection*{run\_pilig\_supfam\_annotate()\label{pibase::SUPFAM_pm_run_pilig_supfam_annotate_}\index{pibase::SUPFAM pm!run\ pilig\ supfam\ annotate()}}
\begin{verbatim}
   Title:       run_pilig_supfam_annotate()
   Function:    Maps PIBASE/LIGBASE binding sites onto target sequences
                  annotated with SUPERFAMILY domain assignments
\end{verbatim}
\begin{verbatim}
   Args:        ->{ARGV} = ARGV array reference; parsed to provide:
                ->{ass_fn} = name of SUPERFAMILY domain assignment file
                ->{out_fn} = name of output file
                ->{err_fn} = name of error file
                ->{matrix_fn} = optional substitution matrix, default BLOSUM62
                ->{cluster_fl} = run on an SGE cluster (options in SGE.pm)
\end{verbatim}
\begin{verbatim}
   Returns:     NOTHING
   Displays:     1. seq_id
                 2. res_range
                 3. classtype
                 4. class
                 5. bs_type
                 6. bs_template
                 7. partner_descr
                 8. residues
                 9. bs_percseqident
                10. bs_percseqsim
                11. bs_numident
                12. bs_numgap
                13. bs_tmpl_numres
                14. bs_fracaln
                15. wholedom_numident
                16. wholedom_aln_length
                17. wholedom_percseqident
                18. wholedom_percseqsim
\end{verbatim}
\subsubsection*{merge\_SUPFAM\_ASTRAL\_alignments\label{pibase::SUPFAM_pm_merge_SUPFAM_ASTRAL_alignments}\index{pibase::SUPFAM pm!merge\ SUPFAM\ ASTRAL\ alignments}}
\begin{verbatim}
   Title:       merge_SUPFAM_ASTRAL_alignments()
   Function:    Merges SUPERFAMILY alignment string with ASTRAL alignment to
                 get SUPERFAMILY annotated target sequence in the ASTRAL
                 alignment frame (where it can receive binding site annotations)
\end{verbatim}
\begin{verbatim}
   Args:        ->{superfam_aln} = SUPERFAMILY alignment string
                ->{astral_aln} = ASTRAL alignment string
                ->{target} = target sequence name
                ->{common} = template sequence name present in both alignments
\end{verbatim}
\begin{verbatim}
   Returns:     ->{resno2alnpos}->{target resno} = alignment position
                ->{alnpos2resno}->{alignment position} = target resno
\end{verbatim}
\subsubsection*{readin\_scop\_cla\label{pibase::SUPFAM_pm_readin_scop_cla}\index{pibase::SUPFAM pm!readin\ scop\ cla}}
\begin{verbatim}
   Title:       readin_scop_cla()
   Function:    Reads in SCOP cla file (parsing logic from pibase::SCOP)
   Args:        ->{fn} = SCOP cla filename
   Returns:     ->{px2scopid}->{px_id} = scopid
                ->{scopid2class}->{scopid} = class
                ->{faid2sfid}->{fa_id} = sf_id0
\end{verbatim}
\subsubsection*{get\_SUPFAM\_selfhit\label{pibase::SUPFAM_pm_get_SUPFAM_selfhit}\index{pibase::SUPFAM pm!get\ SUPFAM\ selfhit}}
\begin{verbatim}
   Title:       get_SUPFAM_selfhit()
   Function:    Retrieves alignment string from SUPFAM self-hit files for
                  a particular SCOP template domain
\end{verbatim}
\begin{verbatim}
   Args:        ->{supfam_specs} = configuration parameters
                ->{px_id} = SCOP px id
                ->{model_id} = SUPERFAMILY model id
\end{verbatim}
\begin{verbatim}
   Returns:     self-hit alignment string
\end{verbatim}
\subsubsection*{summarize\_results()\label{pibase::SUPFAM_pm_summarize_results_}\index{pibase::SUPFAM pm!summarize\ results()}}
\begin{verbatim}
   Title:       summarize_results()
   Function:    Parses run_pilig_supfam_annotate() output and reports
                  summary of annotations.
\end{verbatim}
\begin{verbatim}
   Args:        ->{results_fn} = run_pilig_supfam_annotate() output file
   Returns:     nothing
   Displays:    table describing numbers of proteins,domains,families,residues
                 with each kind of annotation
\end{verbatim}
\clearpage
\section{pibase::aux\label{pibase::aux}\index{pibase::aux}}


Perl interface to auxiliary pibase routines

\subsection*{DESCRIPTION\label{pibase::aux_DESCRIPTION}\index{pibase::aux!DESCRIPTION}}


Perl package that has miscellaneous pibase routines for non-core functions

\subsection*{AUTHOR\label{pibase::aux_AUTHOR}\index{pibase::aux!AUTHOR}}


Fred P. Davis, HHMI-JFRC (davisf@janelia.hhmi.org)

\clearpage
\section{pibase::benchmark\label{pibase::benchmark}\index{pibase::benchmark}}


Perl package for benchmarking pibase

\subsection*{DESCRIPTION\label{pibase::benchmark_DESCRIPTION}\index{pibase::benchmark!DESCRIPTION}}


Contains pibase table structure definitions

\subsection*{AUTHOR\label{pibase::benchmark_AUTHOR}\index{pibase::benchmark!AUTHOR}}


Fred P. Davis, HHMI-JFRC (davisf@janelia.hhmi.org)

\subsection*{SUBROUTINES\label{pibase::benchmark_SUBROUTINES}\index{pibase::benchmark!SUBROUTINES}}
\subsubsection*{memusage()\label{pibase::benchmark_memusage_}\index{pibase::benchmark!memusage()}}
\begin{verbatim}
   Title:       memusage()
   Function:    Returns the VmSize of the process
      works on linux /proc/$$/status
\end{verbatim}
\clearpage
\section{pibase::build\label{pibase::build}\index{pibase::build}}


Perl module to build the pibase database

\subsection*{DESCRIPTION\label{pibase::build_DESCRIPTION}\index{pibase::build!DESCRIPTION}}


Perl module that executes the PIBASE build protocol.

\subsection*{AUTHOR\label{pibase::build_AUTHOR}\index{pibase::build!AUTHOR}}


Fred P. Davis, HHMI-JFRC (davisf@janelia.hhmi.org)

\clearpage
\section{pibase::calc::interfaces\label{pibase::calc::interfaces}\index{pibase::calc::interfaces}}


Perl module to compute protein interfaces.

\subsection*{DESCRIPTION\label{pibase::calc::interfaces_DESCRIPTION}\index{pibase::calc::interfaces!DESCRIPTION}}


Perl module that contains routines for computing structural interfaces

\subsection*{AUTHOR\label{pibase::calc::interfaces_AUTHOR}\index{pibase::calc::interfaces!AUTHOR}}


Fred P. Davis, HHMI-JFRC (davisf@janelia.hhmi.org)

\subsection*{SUBROUTINES\label{pibase::calc::interfaces_SUBROUTINES}\index{pibase::calc::interfaces!SUBROUTINES}}
\subsubsection*{interface\_detect\_calc()\label{pibase::calc::interfaces_interface_detect_calc_}\index{pibase::calc::interfaces!interface\ detect\ calc()}}
\begin{verbatim}
   Title:       interface_detect_calc()
   Function:    Detect interfaces in bdp files.
   Args:        None
   Returns:     Nothing
   STDIN:       bdp_id."\t".bdp_path
   Files in:    PDB files (as specified in STDIN column 2 (bdp_path))
   Files out:
   o intersubset_contacts.<hostname>.<timestamp>.<XXXXXX>.<pibase db name>
   o patch_residues_tables.meta.<hostname>.<timestamp>.<XXXXXX>.<pibase db name>
   o interface_contacts_tables.meta.<hostname>.<timestamp>.<XXXXXX>.
      <pibase db name>
   o interface_contacts_special_tables.meta.<hostname>.<timestamp>.<XXXXXX>.
      <pibase db name>
\end{verbatim}
\begin{verbatim}
   o foreach bdp_id:
      o patch_residues_<bdp_id>.<XXXXXX>.<pibase db name>
      o interface_contacts_<bdp_id>.<XXXXXX>.<pibase db name>
      o interface_contacts_special_<bdp_id>.<XXXXXX>.<pibase db name>
\end{verbatim}
\subsubsection*{\_interface\_detect\_calc\_\_calc\_res\_pairs().\label{pibase::calc::interfaces__interface_detect_calc_calc_res_pairs_}\index{pibase::calc::interfaces!\ interface\ detect\ calc\ \ calc\ res\ pairs().}}
\begin{verbatim}
   Title:       _interface_detect_calc__calc_res_pairs()
   Function:    Calculates residue contacts in a given pdb file.
   Args:        $_->{radius} - upper distance limit on inter-atomic contacts calculation [default=6.6 Ang]
                $_->{compress} - compression flag
                $_->{bdp_path} - bdp file path
\end{verbatim}
\begin{verbatim}
   Return:      $_->{contacts_fn} - kdcontacts output file
                $_->{fields} - kdcontacts file field names
                $_->{field2no}->{field} = i - hash mapping kdcontacts field names to field number
\end{verbatim}
\begin{verbatim}
   Files IN:    PDB file ($_->{bdp_path})
   Files OUT:   kdcontacts output file ($_->{contacts_fn})
\end{verbatim}
\subsubsection*{cluster\_scop\_interfaces()\label{pibase::calc::interfaces_cluster_scop_interfaces_}\index{pibase::calc::interfaces!cluster\ scop\ interfaces()}}
\begin{verbatim}
   Title:       cluster_scop_interfacesj()
   Function:    Clusters SCOP-SCOP interfaces using ASTRAL ASTEROIDS alignments.
                 and imports to PIBASE if specified
   Args:        ->{pibase_specs} = optional
                ->{import_fl} = 1 if to be imported into PIBASE
   Returns:     Prints out cluster membership to
                 $pibase_specs->{buildfiles}->{scop_interface_clusters}) ;
\end{verbatim}
\subsubsection*{\_cluster\_interfaces\_compare\_residue\_sets()\label{pibase::calc::interfaces__cluster_interfaces_compare_residue_sets_}\index{pibase::calc::interfaces!\ cluster\ interfaces\ compare\ residue\ sets()}}
\begin{verbatim}
   Title:       _cluster_interfaces_compare_residue_sets()
   Function:    Venn comparison of interface residue sets 
   Args:        ->{aln} = alignment data
                ->{sid1} = domain 1 identifier
                ->{sid2} = domain 2 identifier
                ->{res1}->{ resno1 => 1, resno2 => 1} - residues in set 1
                ->{res2}->{ resno1 => 1, resno2 => 1} - residues in set 2
   Returns:     union / (union + diff) of the two sets
\end{verbatim}
\subsubsection*{\_cluster\_interfaces\_load\_interface\_contacts()\label{pibase::calc::interfaces__cluster_interfaces_load_interface_contacts_}\index{pibase::calc::interfaces!\ cluster\ interfaces\ load\ interface\ contacts()}}
\begin{verbatim}
   Title:       _cluster_interfaces_load_interface_contacts()
   Function:    Venn comparison of interface residue sets 
   Args:        ->{sid1} = domain 1 identifier
                ->{sid2} = domain 2 identifier
   Returns:     ->{intres}->{sid1}->{resno1."\n".chain1}= domain 1 res in interface
                ->{intres}->{sid2}->{resno2."\n".chain2}= domain 2 res in interface
                ->{contacts}->{resno1."\n".chain1."\n".resno2."\n".chain2} -
                   inter-domain contact
\end{verbatim}
\subsubsection*{\_cluster\_interfaces\_compare\_contact\_sets()\label{pibase::calc::interfaces__cluster_interfaces_compare_contact_sets_}\index{pibase::calc::interfaces!\ cluster\ interfaces\ compare\ contact\ sets()}}
\begin{verbatim}
   Title:       _cluster_interfaces_compare_contact_ses()
   Function:    Venn comparison of interface contact sets 
   Args:        ->{aln1} = alignment data for domain type 1
                ->{aln2} = alignment data for domain type 2
                ->{sid1} = domain 1 identifier
                ->{sid2} = domain 2 identifier
                ->{cont1} = contacts for interface 1
                ->{cont2} = contacts for interface 2
                ->{revfl}->[i] = reversal flag;
                   if 1 switch sid1/2 in ith interface
\end{verbatim}
\begin{verbatim}
   Returns:     union / (union + diff) of contacts
\end{verbatim}
\subsubsection*{\_cluster\_interface\_pibase\_preload()\label{pibase::calc::interfaces__cluster_interface_pibase_preload_}\index{pibase::calc::interfaces!\ cluster\ interface\ pibase\ preload()}}
\begin{verbatim}
   Title:       _cluster_interface_pibase_preload()
   Function:    Preloads PIBASE data necessary for SCOP interface clustering
   Args:        ->{astral} = astral data
                ->{aln2} = alignment data for domain type 2
                ->{sid1} = domain 1 identifier
                ->{sid2} = domain 2 identifier
                ->{cont1} = contacts for interface 1
                ->{cont2} = contacts for interface 2
                ->{revfl}->[i] = reversal flag; if 1 switch sid1/2 in ith interface
\end{verbatim}
\begin{verbatim}
   Returns:     Huge hash of PIBASE data, see code for field explanations
      bdp2contactsfn => $bdp2contactsfn,
      bdp2pdb => $bdp2pdb,
      pdb2bdp => $pdb2bdp,
      bdp_id => $bdp_id,
      sid1 => $sid1,
      sid2 => $sid2,
      class1 => $class1,
      class2 => $class2,
      osid1 => $osid1,
      osid2 => $osid2,
      revfl => $revfl,
      fampairs => $fampairs,
      fampairs_single => $fampairs_single,
      fampairs_single_osid => $fampairs_single_osid,
      pdbchains => $pdbchains,
      chain_2_pdbchain => $chain_2_pdbchain,
      chain_2_start => $chain_2_start,
      chain_2_end => $chain_2_end,
      chain_2_startser => $chain_2_startser,
      chain_2_endser => $chain_2_endser
\end{verbatim}
\clearpage
\section{pibase::create\_raw\_table\_specs\label{pibase::create_raw_table_specs}\index{pibase::create\ raw\ table\ specs}}


Module to create perl code with PIBASE db strx

\subsection*{DESCRIPTION\label{pibase::create_raw_table_specs_DESCRIPTION}\index{pibase::create raw table specs!DESCRIPTION}}


Parses (My)SQL CREATE TABLE statements and displays perl code that defines table structure

\subsection*{AUTHOR\label{pibase::create_raw_table_specs_AUTHOR}\index{pibase::create raw table specs!AUTHOR}}


Fred P. Davis, HHMI-JFRC (davisf@janelia.hhmi.org)

\subsection*{VERSION\label{pibase::create_raw_table_specs_VERSION}\index{pibase::create raw table specs!VERSION}}


' ;

\begin{verbatim}
   print "fpd".pibase::timestamp()."\n\n" ;
\end{verbatim}


print '=head1 DESCRIPTION



The raw\_table\_specs module contains a hard-coded description of the PIBASE
table structures. This file is automatically generated by the
pibase::create\_raw\_table\_specs

\subsection*{AUTHOR\label{pibase::create_raw_table_specs_AUTHOR}\index{pibase::create raw table specs!AUTHOR}}


Fred P. Davis, HHMI-JFRC (davisf@janelia.hhmi.org)

\subsection*{SUBROUTINES\label{pibase::create_raw_table_specs_SUBROUTINES}\index{pibase::create raw table specs!SUBROUTINES}}
\subsubsection*{full\_table\_specs()\label{pibase::create_raw_table_specs_full_table_specs_}\index{pibase::create raw table specs!full\ table\ specs()}}
\begin{verbatim}
   Title:       full_table_specs()
   Args:        none
   Returns:     $->{table_name}->{prikey} = primary key field
                $->{table_name}->{field_name}->[i] = name of ith field
                $->{table_name}->{field_spec}->[i] = type of ith field
\end{verbatim}
\clearpage
\section{pibase::data::access\label{pibase::data::access}\index{pibase::data::access}}


Perl module of pibase data access routines

\subsection*{DESCRIPTION\label{pibase::data::access_DESCRIPTION}\index{pibase::data::access!DESCRIPTION}}


Perl package that provides pibase data access routines

\subsection*{AUTHOR\label{pibase::data::access_AUTHOR}\index{pibase::data::access!AUTHOR}}


Fred P. Davis, HHMI-JFRC (davisf@janelia.hhmi.org)

\clearpage
\section{pibase::data::calc\label{pibase::data::calc}\index{pibase::data::calc}}


Perl module for pibase data calculation routines

\subsection*{DESCRIPTION\label{pibase::data::calc_DESCRIPTION}\index{pibase::data::calc!DESCRIPTION}}


Perl package that provides pibase data calculation routines

\subsection*{AUTHOR\label{pibase::data::calc_AUTHOR}\index{pibase::data::calc!AUTHOR}}


Fred P. Davis, HHMI-JFRC (davisf@janelia.hhmi.org)

\clearpage
\section{pibase::data::external::ASTRAL\label{pibase::data::external::ASTRAL}\index{pibase::data::external::ASTRAL}}


Perl module of ASTRAL data routines

\subsection*{DESCRIPTION\label{pibase::data::external::ASTRAL_DESCRIPTION}\index{pibase::data::external::ASTRAL!DESCRIPTION}}


Perl package that provides interface to ASTRAL data files. Includes routine
to run MAFFT on domain sequences in the case that the ASTRAL alignments
are yet to be released for a new SCOP version.

\subsection*{AUTHOR\label{pibase::data::external::ASTRAL_AUTHOR}\index{pibase::data::external::ASTRAL!AUTHOR}}


Fred P. Davis, HHMI-JFRC (davisf@janelia.hhmi.org)

\clearpage
\section{pibase::data::external::PISA\label{pibase::data::external::PISA}\index{pibase::data::external::PISA}}


Perl interface to PISA routines

\subsection*{DESCRIPTION\label{pibase::data::external::PISA_DESCRIPTION}\index{pibase::data::external::PISA!DESCRIPTION}}


Perl package that provides interface to PISA data routines

\subsection*{AUTHOR\label{pibase::data::external::PISA_AUTHOR}\index{pibase::data::external::PISA!AUTHOR}}


Fred P. Davis, HHMI-JFRC (davisf@janelia.hhmi.org)

\clearpage
\section{pibase::data::external::PQS\label{pibase::data::external::PQS}\index{pibase::data::external::PQS}}


Perl interface to PQS routines

\subsection*{DESCRIPTION\label{pibase::data::external::PQS_DESCRIPTION}\index{pibase::data::external::PQS!DESCRIPTION}}


Perl package that provides interface to PQS data routines

\subsection*{AUTHOR\label{pibase::data::external::PQS_AUTHOR}\index{pibase::data::external::PQS!AUTHOR}}


Fred P. Davis, HHMI-JFRC (davisf@janelia.hhmi.org)

\clearpage
\section{pibase::interatomic\_contacts\label{pibase::interatomic_contacts}\index{pibase::interatomic\ contacts}}


Perl module that deals with interatomic contacts

\subsection*{DESCRIPTION\label{pibase::interatomic_contacts_DESCRIPTION}\index{pibase::interatomic contacts!DESCRIPTION}}


The interatomic\_contacts.pm module deals with queries involving interatomic contacts.

\subsection*{AUTHOR\label{pibase::interatomic_contacts_AUTHOR}\index{pibase::interatomic contacts!AUTHOR}}


Fred P. Davis, HHMI-JFRC (davisf@janelia.hhmi.org)

\subsection*{SUBROUTINES\label{pibase::interatomic_contacts_SUBROUTINES}\index{pibase::interatomic contacts!SUBROUTINES}}
\subsubsection*{contacts\_select()\label{pibase::interatomic_contacts_contacts_select_}\index{pibase::interatomic contacts!contacts\ select()}}
\begin{verbatim}
   Function:    provides a pseudo-mysql select like interface to [gzipped]
                  contacts file in pibase.interatomic_contacts_prototype format
   Args:        $fields - arrayref: ['bdp_id', 'resno_1', 'resno_2', 'distance']
   Return:      results
\end{verbatim}
\subsubsection*{raw\_contacts\_select()\label{pibase::interatomic_contacts_raw_contacts_select_}\index{pibase::interatomic contacts!raw\ contacts\ select()}}
\begin{verbatim}
   Title:       raw_contacts_select()
   Function:    Allows SQL-like SELECT from the raw interatomic contacts
                 tables stored on disk
   Args:        $_[0] = source_file
                $_[1] = SQL-like SELECT query
                $_[2] = params
                  $_[2]->{maxdist} - distance cutoff
   Returns:     $_ - filehandle to of results file
\end{verbatim}
\subsubsection*{contacts\_select\_inter()\label{pibase::interatomic_contacts_contacts_select_inter_}\index{pibase::interatomic contacts!contacts\ select\ inter()}}
\begin{verbatim}
   Title:       contacts_select_inter()
   Function:    Returns inter-domain interatomic-contacts as specified by an
                  SQL-like SELECT query from the raw interatomic contacts tables
                  stored on disk
   Args:        $_[0] = source_file
                $_[1] = SQL-like SELECT query
                $_[2] = resno_2_subset - hash from residue number to domain
                  identifier
                $_[3] = params
                  $_[3]->{maxdist} - distance cutoff
   Returns:     @_ = array of results
\end{verbatim}
\subsubsection*{special\_params()\label{pibase::interatomic_contacts_special_params_}\index{pibase::interatomic contacts!special\ params()}}
\begin{verbatim}
   Title:       special_params()
   Function:    Specifies the parameters (like distance thresholds) for the
                  ``special'' contacts (salt bridges, hydrogen bonds, strong
                  hydrogen bonds, disulfide bonds)
   Args:        $_[0] = source_file
                $_[1] = SQL-like SELECT query
                $_[2] = resno_2_subset - hash from residue number to domain
                  identifier
                $_[3] = params
                  $_[3]->{maxdist} - distance cutoff
   Returns:     @_ = array of results
\end{verbatim}
\subsubsection*{special\_contact()\label{pibase::interatomic_contacts_special_contact_}\index{pibase::interatomic contacts!special\ contact()}}
\begin{verbatim}
   Title:       special_contact()
   Function:    given the distance between a pair of atom/residues, decide
                  whether it meets dist requirements for a hbond, ssbond,
                  or saltbridge
   Args:        $_[0] = contacts information
                 ->{resna1} = name of residue 1
                 ->{atomna1} = name of atom 1
                 ->{resna2} = name of residue 2
                 ->{atomna2} = name of atom 2
                 ->{dist} = distance
                $_[1] = $t - Time::Benchmark timer handle
   Returns:     $_ = contact category (none, salt, ssbond, hbond)
\end{verbatim}
\clearpage
\section{pibase::kdcontacts\label{pibase::kdcontacts}\index{pibase::kdcontacts}}


Package that parses kdcontacts contacts.

\subsection*{DESCRIPTION\label{pibase::kdcontacts_DESCRIPTION}\index{pibase::kdcontacts!DESCRIPTION}}


Parses kdcontacts output and displays it in a tab-delimited format ready for
import into pibase.interatomic\_contacts

\subsection*{AUTHOR\label{pibase::kdcontacts_AUTHOR}\index{pibase::kdcontacts!AUTHOR}}


Fred P. Davis, HHMI-JFRC (davisf@janelia.hhmi.org)

\clearpage
\section{pibase::modeller\label{pibase::modeller}\index{pibase::modeller}}


Module containing routines to call MODELLER for pibase

\subsection*{DESCRIPTION\label{pibase::modeller_DESCRIPTION}\index{pibase::modeller!DESCRIPTION}}


Performs MODELLER operations needed by pibase. (still old-school TOP format)

\subsection*{AUTHOR\label{pibase::modeller_AUTHOR}\index{pibase::modeller!AUTHOR}}


Fred P. Davis, HHMI-JFRC (davisf@janelia.hhmi.org)

\subsection*{SUBROUTINES\label{pibase::modeller_SUBROUTINES}\index{pibase::modeller!SUBROUTINES}}
\subsubsection*{subsets\_2\_modpick()\label{pibase::modeller_subsets_2_modpick_}\index{pibase::modeller!subsets\ 2\ modpick()}}
\begin{verbatim}
   Function:    Converts subsets_details to MODELLER SELECTION SEGMENTS
   Args:        $_[0] = subset_id
                $_[1] = DBI db handle to pibase
   Return:      $_->[] arrayref of MODELLER pick statements to select domain
\end{verbatim}
\subsubsection*{subsetdef\_2\_mod\_pick()\label{pibase::modeller_subsetdef_2_mod_pick_}\index{pibase::modeller!subsetdef\ 2\ mod\ pick()}}
\begin{verbatim}
   Function:    Converts domain definition to MODELLER pick statements
   Args:        $_[0] = arrayref of chain_id
                $_[1] = arrayref of start_resno
                $_[2] = arrayref of end_resno
   Return:      $_->[] arrayref of MODELLER pick statements to select domain
\end{verbatim}
\subsubsection*{get\_salign (modeller\_bin, bdp\_file)\label{pibase::modeller_get_salign_modeller_bin_bdp_file_}\index{pibase::modeller!get\ salign (modeller\ bin, bdp\ file)}}
\begin{verbatim}
   Title:       get_salign()
   Function:    Calls MODELLER.SALIGN to structurally align two pdb files
   Args:        $_->{pdb_fn_1} - name of pdb file 1
                $_->{pdb_fn_2} - name of pdb file 2
                $_->{modeller_bin} - name of MODELLER binary file
\end{verbatim}
\subsubsection*{get\_salign\_seqseq (modeller\_bin, bdp\_file)\label{pibase::modeller_get_salign_seqseq_modeller_bin_bdp_file_}\index{pibase::modeller!get\ salign\ seqseq (modeller\ bin, bdp\ file)}}
\begin{verbatim}
   Title:       get_salign_seqseq()
   Function:    Calls MODELLER.SALIGN to sequence align two pdb files
   Args:        $_->{pdb_fn_1} - name of pdb file 1
                $_->{pdb_fn_2} - name of pdb file 2
                $_->{modeller_bin} - name of MODELLER binary file
\end{verbatim}
\subsubsection*{OLD\_modeller\_subset\_sasa (modeller\_bin, bdp\_file, picks)\label{pibase::modeller_OLD_modeller_subset_sasa_modeller_bin_bdp_file_picks_}\index{pibase::modeller!OLD\ modeller\ subset\ sasa (modeller\ bin, bdp\ file, picks)}}
\begin{verbatim}
   Title:       OLD_modeller_subset_sasa()
   Function:    Calculate the solvent accessible surface area of a pdb file
   Args:        $_[0] = modeller_binary location
                $_[1] = pdb file location
                $_[2] = pick statements for domain
   Returns:     $_[0] = SASA of domain (get_sasa() data structure)
                $_[1] = error_fl - error flag
\end{verbatim}


Specify the temporary alignment TOP file, and the output alignment file.



Generate the actual TOP file.



Specify the location of the MODELLER LOG file.



Run the TOP file through MODELLER.

\subsubsection*{get\_sasa(modeller\_bin, bdp\_file, picks)\label{pibase::modeller_get_sasa_modeller_bin_bdp_file_picks_}\index{pibase::modeller!get\ sasa(modeller\ bin, bdp\ file, picks)}}
\begin{verbatim}
   Title:       get_sasa()
   Function:    parse modeller sasa (psa)
   Args:        $_->{surftyp} = type of MODELLER surface area
                $_->{pdb_fn} = pdb file location
                $_->{modeller_bin} = modeller binary file
   Returns:     $_[0] = results
                $_[1] = resno_rev
                $_[2] = sasa
                $_[3] = error_fl
\end{verbatim}
\subsubsection*{calc\_sasa()\label{pibase::modeller_calc_sasa_}\index{pibase::modeller!calc\ sasa()}}
\begin{verbatim}
   Title:       calc_sasa()
   Function:    Run and parse modeller sasa (psa)
   Args:        $_->{surftyp} = type of MODELLER surface area
                $_->{pdb_fn} = pdb file location
                $_->{modeller_bin} = modeller binary file
   Returns:     ->{res_sasa}->{all_sum|mc_sum|sc_sum|p_sum|nonp_sum}->[i] =
                    SASA information for residue record i
                ->{sasa_resno_rev}->{"residuenumber_chain"} = record number
                ->{full_sasa}->{p|nonp|mc|sc|all} = totals of residue sasa records
                ->{error_fl} => $error_fl,
                ->{atm_sasa}->{p|nonp|mc|sc|all} = totals of ATOM sasa records
\end{verbatim}
\subsubsection*{get\_dihedrals()\label{pibase::modeller_get_dihedrals_}\index{pibase::modeller!get\ dihedrals()}}
\begin{verbatim}
   Title:       get_dihedrals()
   Function:    run and parse modeller dihedrals (dih)
   Args:        $_[0] = bdp_file
                $_[1] = modeller_bin
   Returns:     $_[0] = results
                $_[1] = resno_rev
                $_[2] = error_fl
\end{verbatim}
\subsubsection*{get\_vol()\label{pibase::modeller_get_vol_}\index{pibase::modeller!get\ vol()}}
\begin{verbatim}
   Function:    Run and parse MODELLER volume (psa)
   Args:        $_[0] = bdp_file
                $_[1] = MODELLER binary file
   Returns:     $_[0] = results
                $_[1] = resno_rev
                $_[2] = sasa
                $_[3] = error_fl
\end{verbatim}
\subsubsection*{cutpdb()\label{pibase::modeller_cutpdb_}\index{pibase::modeller!cutpdb()}}
\begin{verbatim}
   Title:       cutpdb()
   Function:    Uses MODELLER to extrct domain from a pdb file
   Args:        $_[0] = MODELLER binary location
                $_[1] = pdb file location
                $_[2] = MODELLER pick statements
                $_[3] = output pdb file name
\end{verbatim}
\begin{verbatim}
   Returns:     $_[0] = 
                $_[1] = resno_rev
                $_[2] = error_fl
\end{verbatim}
\begin{verbatim}
   FILE in:     pdb file ($_[1])
   FILE out:    domain pdb file ($_[3])
\end{verbatim}
\subsubsection*{parse\_ali()\label{pibase::modeller_parse_ali_}\index{pibase::modeller!parse\ ali()}}
\begin{verbatim}
   Title:       parse_ali()
   Function:    reads in a modeller PIR format alignment and returns residue
                number equivalence hashes
   Args:        $_->{ali_fn} alignment file
                $_->{modpipe_newstyle_orderswitch}
                  - 1 (Default) if new style MODPIPE run
                  - reordered sequences in the alignment file
   Results:     ->{seq} = $seq ;
                ->{resno_start} = $resno_start ;
                ->{resno_end} = $resno_end ;
                ->{chain_start} = $chain_start ;
                ->{chain_end} = $chain_end ;
                ->{alipos_2_serial} = $alipos_2_serresno ;
                ->{alipos_2_chainno} = $alipos_2_chainno ;
                ->{alipos_2_resna} = $alipos_2_resna ;
                ->{maxlength} = $maxlength;
\end{verbatim}
\subsubsection*{get\_resequiv()\label{pibase::modeller_get_resequiv_}\index{pibase::modeller!get\ resequiv()}}
\begin{verbatim}
   Title:       get_resequiv()
   Function:    Determines a mapping between residues in two pdb files.
                  Returns a combine serial residue number/positioning from
                  get_resequiv_serial with residue_info()
   Args:        ->{modeller_bin} = MODELLER binary location
                ->{pdb_fn_1} = PDB file 1 location
                ->{pdb_fn_2} = PDB file 2 location
   Returns:     $->[0]->{resno1} = resno2. maps from resno1 in first pdb file
                  to the aligned residue in the second pdb file
                $->[1]->{resno2} = resno1. maps from resno2 in second pdb file
                  to the aligned residue in the first pdb file
\end{verbatim}
\subsubsection*{get\_resequiv\_serial()\label{pibase::modeller_get_resequiv_serial_}\index{pibase::modeller!get\ resequiv\ serial()}}
\begin{verbatim}
   Title:       get_resequiv_serial()
   Function:    Determines residue equivalencies between two pdb files by
                  aligning them structurally using MODELLER.SALIGN
   Args:        ->{modeller_bin} = MODELLER binary location
                ->{pdb_fn_1} = location of pdb file name 1
                ->{pdb_fn_2} = location of pdb file name 2
   Returns:     parse_ali() alignment structure
\end{verbatim}
\clearpage
\section{pibase::pilig\label{pibase::pilig}\index{pibase::pilig}}


Perl module for pibase-ligbase overlap calculations

\subsection*{DESCRIPTION\label{pibase::pilig_DESCRIPTION}\index{pibase::pilig!DESCRIPTION}}


Perl module with routines to cross-query pibase and ligbase
to get small molecule - protein interaction site overlap
statistics

\subsection*{AUTHOR\label{pibase::pilig_AUTHOR}\index{pibase::pilig!AUTHOR}}


Fred P. Davis, HHMI-JFRC (davisf@janelia.hhmi.org)

\clearpage
\section{pibase::pilig\label{pibase::pilig}\index{pibase::pilig}}


Perl module for pibase-ligbase overlap calculations

\subsection*{DESCRIPTION\label{pibase::pilig_DESCRIPTION}\index{pibase::pilig!DESCRIPTION}}


Perl module with routines to cross-query pibase and ligbase
to get small molecule - protein interaction site overlap
statistics

\subsection*{AUTHOR\label{pibase::pilig_AUTHOR}\index{pibase::pilig!AUTHOR}}


Fred P. Davis, HHMI-JFRC (davisf@janelia.hhmi.org)

\clearpage
\section{pibase::pilig\label{pibase::pilig}\index{pibase::pilig}}


Perl module for pibase-ligbase overlap calculations

\subsection*{DESCRIPTION\label{pibase::pilig_DESCRIPTION}\index{pibase::pilig!DESCRIPTION}}


Perl module with routines to cross-query pibase and ligbase
to get small molecule - protein interaction site overlap
statistics

\subsection*{AUTHOR\label{pibase::pilig_AUTHOR}\index{pibase::pilig!AUTHOR}}


Fred P. Davis, HHMI-JFRC (davisf@janelia.hhmi.org)

\clearpage
\section{pibase::raw\_table\_specs- module that specifies pibase table structures\label{pibase::raw_table_specs-_module_that_specifies_pibase_table_structures}\index{pibase::raw\ table\ specs- module that specifies pibase table structures}}




\subsection*{VERSION\label{pibase::raw_table_specs-_module_that_specifies_pibase_table_structures_VERSION}\index{pibase::raw table specs- module that specifies pibase table structures!VERSION}}


fpd100910\_1640

\subsection*{DESCRIPTION\label{pibase::raw_table_specs-_module_that_specifies_pibase_table_structures_DESCRIPTION}\index{pibase::raw table specs- module that specifies pibase table structures!DESCRIPTION}}


The raw\_table\_specs module contains a hard-coded description of the PIBASE
table structures. This file is automatically generated by the
pibase::create\_raw\_table\_specs

\subsection*{AUTHOR\label{pibase::raw_table_specs-_module_that_specifies_pibase_table_structures_AUTHOR}\index{pibase::raw table specs- module that specifies pibase table structures!AUTHOR}}


Fred P. Davis, HHMI-JFRC (davisf@janelia.hhmi.org)

\subsection*{SUBROUTINES\label{pibase::raw_table_specs-_module_that_specifies_pibase_table_structures_SUBROUTINES}\index{pibase::raw table specs- module that specifies pibase table structures!SUBROUTINES}}
\subsubsection*{full\_table\_specs()\label{pibase::raw_table_specs-_module_that_specifies_pibase_table_structures_full_table_specs_}\index{pibase::raw table specs- module that specifies pibase table structures!full\ table\ specs()}}
\begin{verbatim}
   Title:       full_table_specs()
   Args:        none
   Returns:     $->{table_name}->{prikey} = primary key field
                $->{table_name}->{field_name}->[i] = name of ith field
                $->{table_name}->{field_spec}->[i] = type of ith field
\end{verbatim}
\clearpage
\section{pibase::residue\_math\label{pibase::residue_math}\index{pibase::residue\ math}}


Package that handles residue number operations

\subsection*{DESCRIPTION\label{pibase::residue_math_DESCRIPTION}\index{pibase::residue math!DESCRIPTION}}


The pibase::resno module performs common operations on residue numbers.

\subsection*{AUTHOR\label{pibase::residue_math_AUTHOR}\index{pibase::residue math!AUTHOR}}


Fred P. Davis, HHMI-JFRC (davisf@janelia.hhmi.org)

\subsection*{SUBROUTINES\label{pibase::residue_math_SUBROUTINES}\index{pibase::residue math!SUBROUTINES}}
\subsubsection*{residue\_int(resno)\label{pibase::residue_math_residue_int_resno_}\index{pibase::residue math!residue\ int(resno)}}
\begin{verbatim}
   Title:       residue_int()
   Function:    Seperates the integer and insertion code of a residue number
      NOTE: assumes insertion code is always alphanumeric
   Args:        residue number (5 character - number and insertion code)
   Returns:     $_[0] integer portion of the residue number
                $_[1] insertion code of the residue number
\end{verbatim}
\subsubsection*{residue\_add(residue number, increment)\label{pibase::residue_math_residue_add_residue_number_increment_}\index{pibase::residue math!residue\ add(residue number, increment)}}
\begin{verbatim}
   Title:       residue_add()
   Function:    Adds an integer increment to a residue number
   Args:        $_[0] = residue number (full)
                $_[1] = increment
   Returns:     $_ = new residue number
\end{verbatim}
\subsubsection*{residue\_comparison(resno1, resno2)\label{pibase::residue_math_residue_comparison_resno1_resno2_}\index{pibase::residue math!residue\ comparison(resno1, resno2)}}
\begin{verbatim}
   Title:       residue_comparison(resno1, resno2)
   Function:    Compares 2 residues to determine which one is greater.
   Args:        $_[0] = residue number 1
                $_[2] = residue number 2
   Returns:     $_ = comparison result: 0 = equal,
                                        1 = first is greater,
                                        2 = second is greater.
\end{verbatim}
\subsubsection*{residue\_inrange(resno, start, end)\label{pibase::residue_math_residue_inrange_resno_start_end_}\index{pibase::residue math!residue\ inrange(resno, start, end)}}
\begin{verbatim}
   Title:       residue_inrange(resno, start, end)
   Function:    Checks if a residue number is within a range of residue numbers.
   Args:        $_[0] = residue number
                $_[1] = start residue number range
                $_[1] = end residue number range
   Returns:     Comparison result: 0 = out of range, 1 = in range.
\end{verbatim}
\clearpage
\section{pibase::specs\label{pibase::specs}\index{pibase::specs}}


Perl package containing definition of pibase table structures

\subsection*{DESCRIPTION\label{pibase::specs_DESCRIPTION}\index{pibase::specs!DESCRIPTION}}


Contains pibase table structure definitions

\subsection*{AUTHOR\label{pibase::specs_AUTHOR}\index{pibase::specs!AUTHOR}}


Fred P. Davis, HHMI-JFRC (davisf@janelia.hhmi.org)

\subsection*{SUBROUTINES\label{pibase::specs_SUBROUTINES}\index{pibase::specs!SUBROUTINES}}
\subsubsection*{table\_spec(@tablelist)\label{pibase::specs_table_spec_tablelist_}\index{pibase::specs!table\ spec(@tablelist)}}
\begin{verbatim}
   Title:       table_spec()
   Function:    Return mysql DDL format table specs.
   Returns:     $_ hashref pointing from tablename to specs
                  $_->{i} = specs for ith table
\end{verbatim}
\subsubsection*{SUB sql\_table\_spec(@tablelist)\label{pibase::specs_SUB_sql_table_spec_tablelist_}\index{pibase::specs!SUB sql\ table\ spec(@tablelist)}}
\begin{verbatim}
   Title:       sql_table_spec()
   Function:    Return mysql DDL format table specs.
   Args:        $_ hashref pointing from tablename to specs
                $_->{i} = specs for ith table
\end{verbatim}
\clearpage
\section{pibase::tables\_on\_disk\label{pibase::tables_on_disk}\index{pibase::tables\ on\ disk}}


Perl module that provides an SQL like query interface to tables stored on disk.

\subsection*{DESCRIPTION\label{pibase::tables_on_disk_DESCRIPTION}\index{pibase::tables on disk!DESCRIPTION}}


Perl module to interface with tables stored on disk.

\subsection*{AUTHOR\label{pibase::tables_on_disk_AUTHOR}\index{pibase::tables on disk!AUTHOR}}


Fred P. Davis, HHMI-JFRC (davisf@janelia.hhmi.org)

\subsection*{SUBROUTINES\label{pibase::tables_on_disk_SUBROUTINES}\index{pibase::tables on disk!SUBROUTINES}}
\subsubsection*{select\_tod()\label{pibase::tables_on_disk_select_tod_}\index{pibase::tables on disk!select\ tod()}}
\begin{verbatim}
   Title:       select_tod()
   Function:    provides a pseudo-mysql select like interface to [gzipped]
                contacts file in pibase.interatomic_contacts_prototype format
\end{verbatim}
\begin{verbatim}
   Args:        $_[0]- source file
                $_[1]- arrayref ['bdp_id', 'resno_1', 'resno_2', 'distance']
                $_[2]- table name
                $_[3]- where clause
\end{verbatim}
\begin{verbatim}
   Returns:     @_ - array of arrays
                $_[i][j] = jth field of ith result
\end{verbatim}
\clearpage
\section{pibase::web\label{pibase::web}\index{pibase::web}}


Collection of routines for PIBASE web interface

\subsection*{DESCRIPTION\label{pibase::web_DESCRIPTION}\index{pibase::web!DESCRIPTION}}


This module provides routines for the PIBASE web interface.

\subsection*{AUTHOR\label{pibase::web_AUTHOR}\index{pibase::web!AUTHOR}}


Fred P. Davis, HHMI-JFRC (davisf@janelia.hhmi.org)

\subsection*{SUBROUTINES\label{pibase::web_SUBROUTINES}\index{pibase::web!SUBROUTINES}}
\subsubsection*{greeting()\label{pibase::web_greeting_}\index{pibase::web!greeting()}}
\begin{verbatim}
   Title:       greeting()
   Function:    Provides pibase web site greeting
   Args:        $_->{base} [optional] = website base url
   Return:      html code for greeting section of pibase webpage
\end{verbatim}
\subsubsection*{closing()\label{pibase::web_closing_}\index{pibase::web!closing()}}
\begin{verbatim}
   Title:       closing()
   Function:    Provides pibase web site closing
   Args:        $_->{base} [optional] = website base url
   Return:      html code for closing section of pibase webpage
\end{verbatim}
\subsubsection*{find\_object()\label{pibase::web_find_object_}\index{pibase::web!find\ object()}}
\begin{verbatim}
   Title:       find_object()
   Function:    Routine for finding complex, interface, or domain
   Args:        $_->{base} [optional] = website base url
   Return:      STDOUT html page of search results
\end{verbatim}
\subsubsection*{display\_results()\label{pibase::web_display_results_}\index{pibase::web!display\ results()}}
\begin{verbatim}
   Title:       display_results()
   Function:    Formats a list of query results (find_object()) for html viewing
   Args:        $_->{input}->{object_type} = complexes|interfaces|domains
                $_->{details_link}->{object_type} = construct for detail URL link
                $_->{view_link}->{object_type} = construct for viewer URL link
                $_->{results_field}->[i] = ith result field header
                $_->{data}->[i]->[j] = jth field of ith record
                $_->{cgi_h} = CGI handle
                $_->{base} [optional] = website base url
                $_->{basecgi} [optional] = website base url
   Returns:     STDOUT html page of search results
\end{verbatim}
\subsubsection*{get\_object\_details()\label{pibase::web_get_object_details_}\index{pibase::web!get\ object\ details()}}
\begin{verbatim}
   Title:       get_object_details()
   Function:    Formats a list of query results (find_object()) for html viewing
   Args:        $_->{input}->{object_type} = complexes|interfaces|domains
                $_->{details_link}->{object_type} = construct for detail URL link
                $_->{view_link}->{object_type} = construct for viewer URL link
                $_->{results_field}->[i] = ith result field header
                $_->{data}->[i]->[j] = jth field of ith record
                $_->{cgi_h} = CGI handle
                $_->{base} [optional] = website base url
                $_->{basecgi} [optional] = website base url
   Returns:     STDOUT html page of search results
\end{verbatim}
\subsubsection*{foldtext()\label{pibase::web_foldtext_}\index{pibase::web!foldtext()}}
\begin{verbatim}
   Title:       foldtext()
   Function:    Wrapping code for a string with specified folding characters
   Args:        $_->{string} = 'ORIGINALSTRING'
                $_->{wrap} = wrapping length
                $_->{foldchar} = folding charcter [optional - defaults to newline]
   Returns:     $_ = "ORIG\nINAL\nSTRI\nNG" ;
\end{verbatim}
\subsubsection*{get\_color\_codes()\label{pibase::web_get_color_codes_}\index{pibase::web!get\ color\ codes()}}
\begin{verbatim}
   Title:       get_color_codes()
   Function:    Returns rgb decimal and hex values for a specified color name
   Args:        $_->{name} = color name
   Returns:     $_->{rgb} = "R G B" (0-1 scale)
                $_->{rgb255} => "R,G,B" (0-255 scale)
                $_->{hexcode} => "RRGGBB" hex code
\end{verbatim}
\subsubsection*{pngconvert\_bdp\_interaction\_topology\_graph()\label{pibase::web_pngconvert_bdp_interaction_topology_graph_}\index{pibase::web!pngconvert\ bdp\ interaction\ topology\ graph()}}
\begin{verbatim}
   Title:       pngconvert_bdp_interaction_topology_graph()
   Function:    Iterates over all eps format topology graphs and converts to
                 PNG using ImageMagick
   Args:        $_->{pibase_specs} = pibase_specs [optional - if not get_specs()]
   Returns:     Nothing
\end{verbatim}
\subsubsection*{view\_object()\label{pibase::web_view_object_}\index{pibase::web!view\ object()}}
\begin{verbatim}
   Title:       view_object()
   Function:    Retrieves and displays to STDOUT PIBASE structure file for
                 domain, interface, or complex
   Args:        $_->{subset_id} = subset_id of domain
                $_->{subset_id_1} = subset_id of domain 1 of an interface
                $_->{subset_id_2} = subset_id of domain 2 of an interface
                $_->{bdp_id} = bdp_id of a complex
                $_->{subset_source_id} = domain classification system of a complex
   Returns:     STDOUT html page of search results
\end{verbatim}
\subsubsection*{view\_interface\_subset2rasmol()\label{pibase::web_view_interface_subset2rasmol_}\index{pibase::web!view\ interface\ subset2rasmol()}}
\begin{verbatim}
   Title:       view_object()
   Function:    Returns rasmol script commands to define a series of domains
   Args:        $_->{dbh} = PIBASE database handle
                $_->{subsets} = [subset_id_1, 2, ...]
                $_->{colors} = [name of color to use for domain 1, 2, ...]
   Returns:     STDOUT prints rasmol script commands
\end{verbatim}
\clearpage
\section{pibase::web\_pilig\label{pibase::web_pilig}\index{pibase::web\ pilig}}


Collection of routines for PIBASE.ligands web interface

\subsection*{DESCRIPTION\label{pibase::web_pilig_DESCRIPTION}\index{pibase::web pilig!DESCRIPTION}}


This module provides routines for the PIBASE.ligands web interface.

\subsection*{AUTHOR\label{pibase::web_pilig_AUTHOR}\index{pibase::web pilig!AUTHOR}}


Fred P. Davis, HHMI-JFRC (davisf@janelia.hhmi.org)

\subsection*{SUBROUTINES\label{pibase::web_pilig_SUBROUTINES}\index{pibase::web pilig!SUBROUTINES}}
\subsubsection*{greeting()\label{pibase::web_pilig_greeting_}\index{pibase::web pilig!greeting()}}
\begin{verbatim}
   Title:       greeting()
   Function:    Provides pibase web site greeting
   Args:        $_->{base} [optional] = website base url
   Return:      html code for greeting section of pibase webpage
\end{verbatim}
\subsubsection*{closing()\label{pibase::web_pilig_closing_}\index{pibase::web pilig!closing()}}
\begin{verbatim}
   Title:       closing()
   Function:    Provides pibase web site closing
   Args:        $_->{base} [optional] = website base url
   Return:      html code for closing section of pibase webpage
\end{verbatim}
\subsubsection*{find\_object()\label{pibase::web_pilig_find_object_}\index{pibase::web pilig!find\ object()}}
\begin{verbatim}
   Title:       find_object()
   Function:    Routine for finding complex, interface, or domain
   Args:        $_->{base} [optional] = website base url
   Return:      STDOUT html page of search results
\end{verbatim}
\subsubsection*{display\_results()\label{pibase::web_pilig_display_results_}\index{pibase::web pilig!display\ results()}}
\begin{verbatim}
   Title:       display_results()
   Function:    Formats a list of query results (find_object()) for html viewing
   Args:        $_->{input}->{object_type} = complexes|interfaces|domains
                $_->{details_link}->{object_type} = construct for detail URL link
                $_->{view_link}->{object_type} = construct for viewer URL link
                $_->{results_field}->[i] = ith result field header
                $_->{data}->[i]->[j] = jth field of ith record
                $_->{cgi_h} = CGI handle
                $_->{base} [optional] = website base url
                $_->{basecgi} [optional] = website base url
   Returns:     STDOUT html page of search results
\end{verbatim}
\subsubsection*{get\_object\_details()\label{pibase::web_pilig_get_object_details_}\index{pibase::web pilig!get\ object\ details()}}
\begin{verbatim}
   Title:       get_object_details()
   Function:    Formats a list of query results (find_object()) for html viewing
   Args:        $_->{input}->{object_type} = complexes|interfaces|domains
                $_->{details_link}->{object_type} = construct for detail URL link
                $_->{view_link}->{object_type} = construct for viewer URL link
                $_->{results_field}->[i] = ith result field header
                $_->{data}->[i]->[j] = jth field of ith record
                $_->{cgi_h} = CGI handle
                $_->{base} [optional] = website base url
                $_->{basecgi} [optional] = website base url
   Returns:     STDOUT html page of search results
\end{verbatim}
\subsubsection*{foldtext()\label{pibase::web_pilig_foldtext_}\index{pibase::web pilig!foldtext()}}
\begin{verbatim}
   Title:       foldtext()
   Function:    Wrapping code for a string with specified folding characters
   Args:        $_->{string} = 'ORIGINALSTRING'
                $_->{wrap} = wrapping length
                $_->{foldchar} = folding charcter [optional - defaults to newline]
   Returns:     $_ = "ORIG\nINAL\nSTRI\nNG" ;
\end{verbatim}
\subsubsection*{get\_color\_codes()\label{pibase::web_pilig_get_color_codes_}\index{pibase::web pilig!get\ color\ codes()}}
\begin{verbatim}
   Title:       get_color_codes()
   Function:    Returns rgb decimal and hex values for a specified color name
   Args:        $_->{name} = color name
   Returns:     $_->{rgb} = "R G B" (0-1 scale)
                $_->{rgb255} => "R,G,B" (0-255 scale)
                $_->{hexcode} => "RRGGBB" hex code
\end{verbatim}
\subsubsection*{pngconvert\_bdp\_interaction\_topology\_graph()\label{pibase::web_pilig_pngconvert_bdp_interaction_topology_graph_}\index{pibase::web pilig!pngconvert\ bdp\ interaction\ topology\ graph()}}
\begin{verbatim}
   Title:       pngconvert_bdp_interaction_topology_graph()
   Function:    Iterates over all eps format topology graphs and converts to
                 PNG using ImageMagick
   Args:        $_->{pibase_specs} = pibase_specs [optional - if not get_specs()]
   Returns:     Nothing
\end{verbatim}
\subsubsection*{view\_object()\label{pibase::web_pilig_view_object_}\index{pibase::web pilig!view\ object()}}
\begin{verbatim}
   Title:       view_object()
   Function:    Retrieves and displays to STDOUT PIBASE structure file for
                 domain, interface, or complex
   Args:        $_->{subset_id} = subset_id of domain
                $_->{subset_id_1} = subset_id of domain 1 of an interface
                $_->{subset_id_2} = subset_id of domain 2 of an interface
                $_->{bdp_id} = bdp_id of a complex
                $_->{subset_source_id} = domain classification system of a complex
   Returns:     STDOUT html page of search results
\end{verbatim}
\subsubsection*{view\_interface\_subset2rasmol()\label{pibase::web_pilig_view_interface_subset2rasmol_}\index{pibase::web pilig!view\ interface\ subset2rasmol()}}
\begin{verbatim}
   Title:       view_object()
   Function:    Returns rasmol script commands to define a series of domains
   Args:        $_->{dbh} = PIBASE database handle
                $_->{subsets} = [subset_id_1, 2, ...]
                $_->{colors} = [name of color to use for domain 1, 2, ...]
   Returns:     STDOUT prints rasmol script commands
\end{verbatim}
\printindex

\end{document}
