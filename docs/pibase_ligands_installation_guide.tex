\documentclass[11pt]{article}
\usepackage[nohead,margin=2cm,includefoot]{geometry}
\geometry{verbose,letterpaper,tmargin=1in,bmargin=1in,lmargin=1in,rmargin=1in}
\renewcommand{\familydefault}{\sfdefault}
\usepackage{graphicx}
\usepackage{hyperref}
\usepackage{listings}
\title{PIBASE.ligands installation guide. ver 200905\\
\includegraphics*[viewport=469 329 559 415,scale=0.5]{pilig_web_logo.pdf}}
\author{Fred P. Davis, HHMI-JFRC\\{\tt davisf@janelia.hhmi.org}\\\url{http://pibase.janelia.org}}
\begin{document}

\maketitle

\begin{abstract}
This document describes how to set up a local PIBASE.ligands installation.
\end{abstract}

PIBASE.ligands can be installed locally on top of an existing PIBASE (v200808) installation by downloading the database tables from \url{http://pibase.janelia.org/ligands_download.html}. In addition to this MySQL interface, a software package is also available that enables a web interface to the database.

The database schema is described here: \url{http://pibase.janelia.org/files/pibase_schema_v200905.pdf}

\section{Downloading data}

All PIBASE.ligands files are available for download at \url{http://pibase.janelia.org/ligands_download.html} under the GPL license.

\subsection{MySQL data dump}

PIBASE.ligands installation {\em requires} a working PIBASE installation ( \url{http://pibase.janelia.org/files/pibase_installation_guide.pdf} ). Once you have the PIBASE database installed, download the PIBASE.ligands MySQL dump ( pibase\_ligands\_dump.20090519.out.gz ) and load it into the PIBASE database:

\begin{verbatim}
zcat pibase_ligands_dump.20090518.out.gz | mysql -u YOURUSERNAME -p YOURDATABASENAME
\end{verbatim}

\section{Web interface}

To install the web interface to the PIBASE.ligands database, you must first have a working PIBASE web server installed. Once you have the PIBASE web interface installed, download the PIBASE.ligands web server package (pibase\_ligands\_src\_v200905.tar.gz) and uncompress it in a temporary directory:

\begin{verbatim}
cd yourinstalldirectory
tar xvfz pibase_ligands_src_200905.tar.gz
\end{verbatim}

Next, there are a few lines to edit in pibase.pm to reflect your database specifications and local directory structure. Then, just copy over the html, cgi-bin, and perl library to your webserver, and it should be ready to query.

In detail:

\begin{enumerate}
\item Edit pibase.pm to reflect your MySQL database speccs. ($~$ src/perl\_api/pibase.pm lines 57-60).

\begin{verbatim}
my $pibase_specs = {
   db => 'pibasemysqldatabasename',
   host => 'mysqlserverhostname',
   user => 'pibaseusername' ,
   pass => 'pibasepassword',
   root => 'doesntmatterforthewebserver',
}
\end{verbatim}

If you have a non-standard mysql installation, you can also specify a {\tt mysql\_socket} key that points to the mysql socket to be used by the perl DBI mysql interface.

\item Edit pibase.pm to reflect the html and cgi-bin directories of your web server. ($~$ src/perl\_api/pibase.pm lines 216, 217).
\begin{verbatim}
   $pibase_specs->{web_pilig}->{html_dir} = "/var/www/websites/pibase/html/" ;
   $pibase_specs->{web_pilig}->{cgi_dir} = "/var/www/websites/pibase/cgi-bin/" ;
\end{verbatim}

\item Edit pibase.pm to reflect the correct URLS for the html and cgi-bin directories of your web server. ($~$ src/perl\_api/pibase.pm lines 219, 220).
\begin{verbatim}
   $pibase_specs->{web_pilig}->{base_url} = "http://localhost/pibase";
   $pibase_specs->{web_pilig}->{basecgi_url} = "http://localhost/pibase-cgi";
\end{verbatim}

\item Edit pibase.pm to reflect the directories where you uncompressed the bdp\_topology\_graphs and subsets\_files tar files for the original PIBASE web server. ($~$ src/perl\_api/pibase.pm lines 209, 213).
\begin{verbatim}
   $pibase_specs->{web_pilig}->{bdp_topology_graphs_baseurl} =
      $pibase_specs->{web_pilig}->{base_url}.'/data_files/bdp_topology_graphs' 

   $pibase_specs->{web_pilig}->{subsets_files_basedir} =
      $pibase_specs->{web_pilig}->{html_dir}.'/data_files/subsets_files' ;
\end{verbatim}

\item The static html pages (web/html/*.html) assume that the CGI directory sits ../cgi-bin relative to the html directory; if this isn't true for your webserver, edit the form submit lines in the html pages to reflect the path to your CGI directory

\item Lastly, copy the contents of the html, cgi, and perl\_api directories to your webserver
\begin{verbatim}
   cp -r web/html/* yourwebserver/html
   cp -r web/cgi-bin/* yourwebserver/cgi-bin
   cp -r src/perl_api yourwebserver/cgi-bin/perl_lib
\end{verbatim}
\end{enumerate}

That should be it, the webserver should be functional now.

\end{document}
